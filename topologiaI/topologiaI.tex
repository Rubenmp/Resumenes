\documentclass[11pt,spanish]{article} % Idioma
\usepackage{babel}
\usepackage[T1]{fontenc}
\usepackage{textcomp, verbatim} % \begin{comment}
\usepackage[utf8]{inputenc} % Permite acentos

\usepackage{wrapfig} % Imagenes %\graphicspath{ {./imagenes/} }
\usepackage[left=2.75cm,top=2.5cm,right=2cm,bottom=2.5cm]{geometry} % Márgenes
\usepackage{amssymb, amsmath, amscd, amsfonts, amsthm, mathrsfs } % Símbolos matemáticos
\usepackage{cancel} % Cancelar expresiones
\usepackage{multirow, multicol, tabularx, booktabs, longtable} % Tablas
\usepackage{fancyhdr, fncychap} % Encabezados
\usepackage{algpseudocode, algorithmicx, algorithm} % Pseudo-código	
\usepackage{bbding} % Símbolos
\usepackage{enumitem} % Enumerados a), b), c)... usando \begin{enumerate}[label=\alph*)]
\usepackage{graphicx, xcolor, color, pstricks} % Gráficos --TikZ-- 
% http://www.texample.net/tikz/examples/
\usepackage[hidelinks]{hyperref}  % Enlaces
\usepackage{verbatim} % Comentarios largos \begin{comment}
\usepackage{rotating} % \begin{rotate}{30}
\usepackage[all]{xy} % Diagramas
\usepackage{listings} % Escribir código 
\usepackage{xparse} % Entornos


% Comandos
\newcommand{\docdate}{}
\newcommand{\subject}{}
\newcommand{\docauthor}{Rubén Morales Pérez}
\newcommand{\docemail}{srmorales@correo.ugr.es}

\newcommand{\N}{\mathbb{N}}
\newcommand{\Q}{\mathbb{Q}}
\newcommand{\C}{\mathbb{C}}
\newcommand{\R}{\mathbb{R}}
\newcommand{\Z}{\mathbb{Z}}


\linespread{1.1}                  % Espacio entre líneas.
\setlength\parindent{0pt}         % Indentación para párrafo.

\title{}
\author{ }
\date{\today}

% % % % % % % % % % % % % % % % % % % % % % % % % % % % % % % % %
%					 Inicio del documento
% % % % % % % % % % % % % % % % % % % % % % % % % % % % % % % % %
\begin{document}

\maketitle
\setlength\parindent{0pt} % Quitamos la sangría


\section{Espacios topológicos}
Un espacio topológico es un par $(X,\tau)$ con $X\not =\emptyset$, topología $\tau\subseteq P(X)$, $O\in\tau$ son abiertos y
\begin{itemize}
	\item $\emptyset,X\in\tau$
	\item $O_1,O_2\in\tau \implies O_1\cap O_2\in\tau$
	\item $\{O_{\lambda}, \lambda\in\Lambda\} \subseteq\tau \implies \cup_{\lambda\in\Lambda} O_{\lambda}\in\tau$	
\end{itemize}
\begin{center}
Discreta $X\not =\emptyset$, $\tau_d :=P(X)\hspace{1cm}$

Fuerte $X\not =\emptyset$, $p_O\in X$ fijo, 
$\tau_f:= \{O\in X : p_0\not\in O\text{ o } X-O \text{ finito} \}$

Sierpinski $X=\{a,b\}$, $\tau := \{\emptyset,\{a\},X\} \hspace{1cm}$

Sorgenfrey $(\R,\tau_s)$, $O\in\tau_s \Longleftrightarrow \ \forall x\in O, \exists\epsilon>0 : [x,x+\epsilon[ \subseteq O$
\end{center}

\input{./Temas/2Aplicaciones.tex}
\input{./Temas/3.1Conexion.tex}
\input{./Temas/3.2Compacidad.tex}
%%%%%%%%%%%%%%%%%%%%%%%%%%%%%%%%%%%%%%%%%%%%%%%%%%%%%%%%%%%%%%%%%%%%%%%%%%%%%%%%%%1

% % % % % % % % % % % % % % % % % % % % % % % % % % % % % % % % %
%					 Bibliografía
% % % % % % % % % % % % % % % % % % % % % % % % % % % % % % % % %

\end{document}
