\section{Sucesiones monótonas}
Una sucesión $\{x_n\}$ es creciente (resp. decreciente) si $x_n\leq x_{n+1} \text{(resp. $x_n\geq x_{n+1} $)}\ \ \forall n\in\N$. Es monótona si es creciente o decreciente.
Una sucesión creciente está minorada, una decreciente está mayorada.

\begin{itemize}
	\item Toda sucesión monótona y acotada es convergente, a su supremo o  ínfimo.
	\item Toda sucesión de números reales admite una sucesión parcial monótona.
\end{itemize}

\textbf{Teorema de Bolzano-Weierstrass}

Toda sucesión acotada de números reales admite una sucesión parcial convergente.

\subsection{Sucesiones de Cauchy}
\begin{center}
$ \forall\epsilon >0, \ \exists m\in\N \ : \ p,q\geq m \implies |x_p-x_q| < \epsilon $
\end{center}
$$ \text{Sucesión convergente} \implies \text{Sucesión de Cauchy (}\Longleftarrow \text{Complitud de $\R$)} $$

\subsection{Límites superior e inferior}
\begin{center}
$ \alpha_n = \inf\{x_k \ : \ k\geq n\} \text{ creciente y mayorada}
\hspace{1cm}
\beta_n  = \sup\{x_k \ : \ k\geq n\} \text{ decreciente y minorada} $
\end{center}

Los límites de dichas sucesiones son los límites inferior y superior. Son equivalentes:
\begin{center}
	 $\liminf \{x_n\} = \limsup \{x_n\} 
	\Longleftrightarrow
	\{x_n\} \text{ es convergente}$
\end{center}
$$ \liminf \{x_n\} \leq \lim \{x_{\sigma (n)}\} \leq \limsup \{x_n\} $$
\textbf{Teorema de Bolzano-Weierstrass} (revisitado)

Si $\{x_n\}$ es una sucesión acotada de números reales, entonces admite dos sucesiones parciales tales que
$$ \{x_{\sigma (n)}\} \rightarrow \liminf \{x_n\} \hspace{1cm}
   \{x_{\tau   (n)}\} \rightarrow \limsup \{x_n\}$$

