\section{Supremo e ínfimo. Números irracionales}
\subsection{Supremo e ínfimo}
Dado $\emptyset\not = A\subset \R$, $A$ está mayorado si existe $y\in\R$ llamado mayorante tal que $y\geq a \ \forall a\in A$. Analogamente con minorado y minorante.
 Mayorantes y minorantes no siempre son del conjunto.

Si $A$ está mayorado y minorado está acotado.
Si el conjuntos de mayorantes tiene mínimo, es el supremo, si $A$ está minorado, el máximo de los minorantes es el ínfimo.
$$ a\leq y \ \ \forall a\in A \Longleftrightarrow sup(A)\leq y
\hspace{2cm}
x\leq a \ \ \forall a\in A \Longleftrightarrow x \leq inf(A)$$

\textbf{Relación con máximo y mínimo (
$\emptyset \not = A \subset \R , \ \alpha\in\R$)}
$$ \alpha = max (A) \Longleftrightarrow \alpha\in A, \ a\leq\alpha \ \ \forall a\in A 
\hspace{1cm}
\alpha = sup (A) \Longleftrightarrow 
  \left\lbrace
  \begin{array}{l}
	  a\leq\alpha \ \  \forall a\in A \\
	  \forall\epsilon\in\R ^+ , \ \exists a\in A \ : \ \alpha - \epsilon < a\\
  \end{array}
  \right.
$$

\textbf{Raíz n-ésima.}
Dado $n\in\N$, para cada $x\in\R ^+$ existe un único $y\in\R ^+$ tal que $y^n=x$, $y=\sqrt[n]{x}$
Dados $n,m\in\N$, $\sqrt[n]{m}$ es un número natural o un número irracional.

\subsection{Propiedad arquimediana}
$\N$ no está mayorado, $\forall x\in\R ,\ \exists n\in\N \ : \ n>x$.
Sea $\emptyset\not =A\subset\Z$
\begin{itemize}
	\item Si $A$ está mayorado(minorado/acotada), entonces $A$ tiene máximo(mínimo/es finito).
\end{itemize}
\textbf{Función parte entera.} 
$E(x) = max\{k\in\Z \ : \ k\leq x\}. \hspace{0.5cm} \forall x\in\R$, $E(x)\leq x\leq E(x)+1$

\subsection{Densidad de $\Q$ y $\R\backslash\Q$ en $\R$}
Un conjunto $D\subset \R$ es denso en $\R$ cuando, $\forall x,y\in\R \ : \ x<y, \ \exists d\in D \ : \ x<d<y$

Para cualesquiera $x,y\in\R$ con $x<y$, existen $r\in\Q, \ \beta\in\R\backslash\Q$ verificando que $x<r<\beta <y$
\begin{center}
$\forall x\in\R ,  \ 
sup\{ r\in\Q \ : \ r<x \} = x = inf\{ s\in\Q \ : \ s>x \}$	
\end{center}


\subsection{Intervalos}
$I$ es un intervalo si $\forall x,y\in I : x<y$, entonces $\forall z\in\R$ con $x<z<y\implies z\in I$. La intersección de cualquier familia de intervalos es un intervalo. Todo intervalo no trivial es no numerable.

\textbf{Principio de los intervalos encajados} (Georg Cantor):

Suponemos que, para cada $n\in\N$ tenemos un intervalo cerrado y acotado $J_n=[a_n,b_n]$, con $a_n\leq b_n$, y cada uno de estos intervalos contiene al siguiente ($J_{n+1}\subset J_n$) para todo $n\in\N$. Entonces la intersección $\bigcap _{n\in\N}J_n$ no es vacía, es decir, existe $x\in\R$ tal que $x\in J_n$ para todo $n\in\N$.

\subsection{Números alebraicos y trascendentes}
Un número real es algebraico cuando se puede obtener como solución de una ecuación algrebraica con coeficientes enteros.
Los números irracionales que hasta ahora conocemos son algebraicos. El conjunto de los números algebraicos es numerable.
Un número real es trascendente cuando no es algebraico.