\section{Números naturales, enteros y racionales}
$\N$ es la intersección de todos los subconjuntos inductivos de $\R$. $A\subset \R$ es inductivo si verifica:
$$ 1\in A \hspace{2cm}
x\in A \implies x+1\in A$$



\subsection{Principio de inducción}
Si $A$ es un subconjunto de $\N$ y $A$ es inductivo, entonces $A=\N$

\subsection{Principio de buena ordenación de los números naturales}
Todo conjunto no vacío de números naturales tiene mínimo.

Un conjunto bien ordenado está ordenado y todo subconjunto no vacío suyo tiene mínimo.

\subsection{Potencias de una suma y sumas de potencias}
\begin{center}
	\textbf{Números combinatorios ($n\in\N, \ n! =n\cdot(n-1)\cdot\hdots\cdot2\cdot 1$)}
\end{center}
$$ \binom{n}{k} = 			
	\frac{n!}{k!(n-k)!}
\hspace{2cm}
   \binom{n}{0} = \binom{n}{n} = 1   
\hspace{2cm} 
 \binom{n+1}{k} = \binom{n}{k} + \binom{n}{k-1} 
$$ 
\begin{center}
	\textbf{Binomio de Newton}
\end{center}
$$(x+y)^n = \sum_{k=0}^{n} \binom{n}{k} x^{n-k}y^k$$

\begin{center}
	\textbf{Suma de una progresión geométrica} 
\end{center}
$$\sum_{k=0}^n z^k= \frac{z^{n+1} - 1}{z-1} \hspace{1cm} \forall z\in\R \textbackslash \{1\}, \  \forall n \in\N$$
$$ \frac{x^{n+1}-y^{n+1}}{x-y} = \sum_{k=0}^n x^{n-k}y^k
 \hspace{1cm} \forall x,y\in\R, \ \forall n\in\N $$
 
\textbf{Grupo abeliano: }
Conjunto con operación asociativa y conmutativa,
 elementos neutro y simétrico.