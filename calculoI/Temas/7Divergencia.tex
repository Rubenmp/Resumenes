\section{Divergencia de sucesiones}
La divergencia no es lo contrario de convergencia, véase 
$\{ (-1)^n \}$. 
Una sucesión $\{x_n\}$ no acotada no implica $\{1/x_n\} \rightarrow 0$, ejemplo
$$ y_n = n+(-1)^n+1 \implies \{1/y_n\}\not\rightarrow 0 \text{ ya que } \{1/y_{2n-1}\}\rightarrow 1 $$
Divergencia positiva equivale a que 
$\{n\in\N \ : \ x_n\leq K\in\R \}$ sea finito.
$$ \{x_n\}\rightarrow +\infty \Longleftrightarrow
 \left[ \forall K\in\R, \ \exists m\in\N \ : n\geq m \implies \ x_n> K \right]$$
$$ \{ |x_n| \} \rightarrow +\infty \Longleftrightarrow \{ x_n \} \rightarrow \infty$$
Que una sucesión diverja no significa que lo haga positiva o negativamente. Ejemplo $\{ (-1)^nn \}$

\subsection{Relación con otras sucesiones}
Toda sucesión parcial de una sucesión divergente es divergente.
Son equivalentes
\begin{itemize}
	\item $\{ x_n \}$ no es divergente
	\item $\{ x_n \}$ admite una sucesión parcial acotada
	\item $\{ x_n \}$ admite una sucesión parcial convergente
\end{itemize} 
Podemos encontrar una sucesión no mayorada, ni minorada y no divergente: 
$$y_{3k-2}=k, \ y_{3k-1}=-k ,\  y_{3k}=0 \hspace{1cm} \underbrace{1,-1,0},\underbrace{2,-2,0},\underbrace{3,-3,0,}...$$
Toda sucesión monótona es convergente o divergente.

\subsection{Operaciones}
$$ \{ x_n \} \text{ e } \{ y_n \} \text{ convergentes} \implies \{ x_n+y_n \} \text{ convergente } \hspace{1cm}\text{(Recíproco falso)}  $$
$$ \text{Si } \{ x_n \}\rightarrow +\infty \text{ e } \{ y_n \} \text{ está minorada en }\R^+ \text{, entonces } \{ x_n+y_n \}\rightarrow +\infty, \ \{x_ny_n\}\rightarrow +\infty $$

Toda sucesión puede expresarse como suma de dos divergentes, o como multiplicación de una que diverja positivamente y otra que converja a 0.
$$ \{ x_n \}\rightarrow\infty,\ \{ y_n \}\rightarrow\lambda\in\R^* \implies \{ x_ny_n \}\rightarrow\infty $$
$$\text{Si } x_n\in\R^+, \ \forall n\in\N\text{, entonces } \{ x_n \}\rightarrow 0 \Longleftrightarrow \{ 1/x_n \} \text{ es divergente} $$
Sea $x_n\in\R^+_0, \ \forall n\in\N$ y sea $q\in\N$
\begin{itemize}
	\item $\{ x_n \}\rightarrow x\in\R^+_0 \implies \{ \sqrt[q]{x_n} \}\rightarrow \sqrt[q]{x}$
	\item $\{ x_n \}\rightarrow +\infty \implies \{ \sqrt[q]{x_n} \}\rightarrow +\infty$	
\end{itemize}