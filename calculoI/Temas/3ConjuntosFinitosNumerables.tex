\section{Conjuntos finitos y conjuntos numerables}
\subsection{Conjuntos finitos}
$A,B$ conjuntos, $A$ es equipotente a $B$, ó $A\sim B$ cuando
existe una aplicación biyectiva de $A$ sobre $B$. 

Si $A$ es finito (biyectivo con $\{k\in\N : k\leq n\}$)  $B$ también lo es si existe una inyección de $B$ en $A$ o una sobreyección de $A$ en $B$.
La imagen de un conjunto finito por cualquier aplicación es un conjunto finito.
Todo conjunto no vacío y finito de números reales tiene máximo y mínimo.

\subsection{Conjuntos numerables}
$A$ es numerable si $A=\emptyset$ o existe una aplicación inyectiva de $A$ en $\N$. 
Si $A$ es un subconjunto infinito de $\N$, existe una aplicación biyectiva $\sigma : \N \rightarrow A$, que tiene la siguiente propiedad:
$$ n,m\in\N, \ n<m \ \implies \ \sigma (n) < \sigma (m) $$

\subsubsection{Ejemplos}
Si $A$ es un conjunto numerable y $f:A \rightarrow C$ una aplicación, entonces $f(A)$ es numerable.

\textbf{Producto cartesiano.} $A,B$ conjuntos numerables $\implies \ AxB$ es numerable

La unión numerable de conjuntos numerables es numerable. (Ejemplo: $\Z$)

$\Q$ es numerable, imagen de un conjunto numerable por una aplicación
$$ f(p,m) = \frac{p}{m} \hspace{1cm} \forall p\in\Z , \ \forall m\in\N$$ 