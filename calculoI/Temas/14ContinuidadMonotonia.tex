\section{Continuidad y monotonía}
Una función es estrictamente creciente si, y sólo si, es creciente e inyectiva

\textbf{Teorema.} Sea $I$ un intervalo y $f:I\rightarrow\R$ continua e inyectiva, entonces $f$ es estrictamente monótona.

\textbf{Teorema.} Si $f:A\rightarrow\R$ es monótona y $f(A)$ es un intervalo, entones $f$ es continua.

La inversa de una función se obtiene  mediante la simetría con eje la recta $y=x$.
Existen funciones continuas e inyectivas cuya inversa no es continua.

\begin{itemize}
	\item Si $f:A\rightarrow\R$ es estrictamente creciente, su inversa también lo es
	\item Si $I$ es un intervalo y $f:I\rightarrow\R$ es estrictamente monótona, entonces $f^{-1}$ es continua
	\item Si $I$ es un intervalo y $f:I\rightarrow\R$ es continua e inyectiva, entonces $f^{-1}$ es continua
\end{itemize}