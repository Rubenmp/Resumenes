\section{Series de números reales}
\begin{center}
	$ S_n = \sum_{k=1}^{n} x_k \hspace{1cm}\text{Suma parcial de la serie}$
\end{center}
El término general de una serie convergente es una sucesión convergente a cero.
\begin{itemize}
	\item La serie armónica con exponente $p$ $\sum_{n\geq 1} \frac{1}{n^p}$ converge para $p=1$ y diverge para $p\in\N\backslash\{1\}$
	\item La serie geométrica con razón $r \ : |r|<1 \ $ es $\sum_{n=0}^{\infty} r^n = \frac{1}{1-r}$
	\item Suma de los términos de una progresión geométrica con términos $a_n \ \forall n\in\N$, $\frac{a_{n+1}-a_1}{r-1}$
	\item Fijando $p\in\N$ una seria converge si, y sólo si, la serie omitiendo los primeros $p$ términos converge
\end{itemize}