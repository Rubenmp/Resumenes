\documentclass[11pt,spanish]{article} % Idioma
\usepackage{babel}
\usepackage[T1]{fontenc}
\usepackage{textcomp, verbatim} % \begin{comment}
\usepackage[utf8]{inputenc} % Permite acentos

\usepackage{wrapfig} % Imagenes %\graphicspath{ {./imagenes/} }
\usepackage[left=2.75cm,top=2.5cm,right=2cm,bottom=2.5cm]{geometry} % Márgenes
\usepackage{amssymb, amsmath, amscd, amsfonts, amsthm, mathrsfs } % Símbolos matemáticos
\usepackage{cancel} % Cancelar expresiones
\usepackage{multirow, multicol, tabularx, booktabs, longtable} % Tablas
\usepackage{fancyhdr, fncychap} % Encabezados
\usepackage{algpseudocode, algorithmicx, algorithm} % Pseudo-código	
\usepackage{bbding} % Símbolos
\usepackage{enumitem} % Enumerados a), b), c)... usando \begin{enumerate}[label=\alph*)]
\usepackage{graphicx, xcolor, color, pstricks} % Gráficos --TikZ-- 
% http://www.texample.net/tikz/examples/
\usepackage[hidelinks]{hyperref}  % Enlaces
\usepackage{verbatim} % Comentarios largos \begin{comment}
\usepackage{rotating} % \begin{rotate}{30}
\usepackage[all]{xy} % Diagramas
\usepackage{listings} % Escribir código 
\usepackage{xparse} % Entornos


% Comandos
\newcommand{\docdate}{}
\newcommand{\subject}{}
\newcommand{\docauthor}{Rubén Morales Pérez}
\newcommand{\docemail}{srmorales@correo.ugr.es}

\newcommand{\N}{\mathbb{N}}
\newcommand{\Q}{\mathbb{Q}}
\newcommand{\C}{\mathbb{C}}
\newcommand{\R}{\mathbb{R}}
\newcommand{\Z}{\mathbb{Z}}


\linespread{1.1}                  % Espacio entre líneas.
\setlength\parindent{0pt}         % Indentación para párrafo.

\title{Análisis I}
\author{ }
\date{ }

% % % % % % % % % % % % % % % % % % % % % % % % % % % % % % % % %
%					 Inicio del documento
% % % % % % % % % % % % % % % % % % % % % % % % % % % % % % % % %
\begin{document}

\maketitle
\setlength\parindent{0pt} % Quitamos la sangría


\section{El espacio euclídeo, espacios normados y espacios métricos}

$\R^n = \R \text{x}\R \text{x }\overset{(N)}{\cdots} \text{ x}\R$ con la suma y producto por escalares tiene estructura de espacio vectorial. La base usual es $\Phi =\{e_k : k\in I_N\}$ con $e_k(k)=1$ y $e_k(j) = 0 \ \forall j\in I_N\backslash \{k\}$.

El producto escalar de dos vectores $x,y\in \R^N$ es $(x|y) = \sum_{k=1}^{N} x(k) y(k)$ cumple:
\begin{itemize}
	\item \textbf{(P.1)} $(\lambda u+\mu v) = \lambda (u|y) +\mu (v|y) \hspace{0.5cm} \forall u,v,y\in\R^N, \ \forall \lambda,\mu\in\R$
	\item \textbf{(P.2)} $(x|y) = (y|x) \hspace{0.5cm} \forall x,y\in\R^N$
	\item \textbf{(P.3)} $(x|x)>0 \hspace{0.5cm} \forall x\in\R^N\backslash\{0\}$
\end{itemize}
$\phi$ es una forma bilineal cuando es lineal en cada variable, simétrica si $\phi(x,y) = \phi(y,x)$. La forma cuadrática asociada se define como $Q(x) = \phi(x,x)$.
Un espacio pre-hilbertiano es un espacio vectorial con un producto escalar. 

La norma de un vector $x\in X$ es $||x|| = (x|x)^{1/2}$, intuitivamente es la longitud del vector. Cumple:
\begin{itemize}
	\item \textbf{(N.1)} $ ||x+y|| \leq ||x||+||y||\hspace{0.5cm} \forall x,y\in X \hspace{2cm} \text{(desigualdad triangular)}$
	\item \textbf{(N.2)} $||\lambda x|| = |\lambda| \ ||x|| \hspace{0.5cm} \forall x\in X,\ \forall\lambda\in\R \hspace{2cm}(homogeneidad por homotecias)$
	\item \textbf{(N.3)} $x\in X, \ ||x||=0 \implies x=0 \hspace{3cm} \text{(no degeneración)}$
\end{itemize}
\textbf{Desigualdad de Cauchy-Schwartz.} En todo espacio pre-hilbertiano X
\begin{center}
$ |(x|y)| \leq ||x|| \ ||y||\hspace{0.5cm}\forall x,y\in X
\hspace{2cm} \text{Igualdad sii $x,y$ linealmente dependientes}  $
\end{center}

Un espacio normado es un espacio vectorial con una norma $||\cdot||$. Todo espacio pre-hilbertiano es normado. 
Una norma puede no proceder de un producto escalar, ejemplo de ello son las normas del máximo y de la suma, con ellas $\R^N$ es espacio normado, no pre-hilbertiano.
$$ ||x||_{\infty} =\max \{|x_k| : k\in \{1,2,\cdots,N\} \hspace{2cm} ||x||_1 = \sum_{k=1}^{N} |x_k| $$

\subsection{Ortogonalidad/perpendicularidad}
$$ x\perp y \overset{\vartriangle}{\Longleftrightarrow} 
(x|y) = 0 \Longleftrightarrow
|| x+y ||^2 = ||x||^2 + ||y||^2$$
Un conjunto $A$ es ortogonal si $\forall x,y\in A, \ x\not =y, x\perp y$, se llama ortonormal si además $||x||=1 \ \forall x\in A$
$$ \text{Ángulo entre vectores no nulos } \alpha (x,y)=\arccos \frac{(x|y)}{||x||\ ||y||}\in[0,\pi]  $$

En un espacio normado se define la distancia por $d(x,y) = ||y-x||$. En cualquier conjunto definimos la distancia discreta como 
$$ \sigma (x,y) = 
\left\{ 
\begin{array}{l}
	0 \text{ si } x=y \\
	1 \text{ si } x\not =y \\
\end{array} \right.$$
\section{Topología de un espacio métrico}
Bola abierta $B(x,r) = \{y\in E : d(x,y)<r \} = \{x+ru : u\in B(0,1) \}$, su aspecto cambio cuando utilizamos una norma distinta de la euclídea.
Los abiertos de un espacio métrico son las uniones de bolas abiertas.
Dos distancias son equivalentes cuando generan la misma topología, dos normas lo son si lo son sus distancias asociadas.
Para dos normas $||\cdot||_1$ y $||\cdot||_2$ en un espacio vectorial equivalen:
\begin{itemize}
	\item $\exists p\in\R^+ : ||x||_2\leq p||x||_1 \ \forall x$
	\item La topología de la norma $||x||_2$ está incluida en la de $||x||_1$
\end{itemize}
Todas las normas en $\R^N$ son equivalentes.
Si $\tau$ es una topoloía en $E$ y $\tau_A$ la de un subespacio métrico
$$ \tau_A = \{ U\cap A : U\in\tau \} \hspace{2cm}
A^o = \bigcup \{U\in\tau : U\subset A \hspace{2cm}
\overline{A} = \bigcap \{ C\in\mathcal{C} : A\in\mathcal{C} \}$$
EL interior es el máximo abierto incluido en $A$, un conjunto es abierto si, y sólo si, es entorno de todos sus puntos. Un conjunto es denso en $\R$ si su adherencia es $\R$
$$ x\in A^o \Longleftrightarrow A\in\mathcal{U}(x)
\Longleftrightarrow \exists\epsilon>0 : B(x,\epsilon)\subset A $$
$$ x\in\overline{A}\text{ punto adherente } \Longleftrightarrow
U\cap A \not =0 \hspace{0.5cm} \forall U\in\mathcal{U}(x)\Longleftrightarrow
B(x,\epsilon)\cap A \not=0 \hspace{0.5cm} \forall\epsilon\in\R^+ $$
$$ E\backslash \overline{A} = (E\backslash A)^o \hspace{2cm}
 E\backslash A^o = \overline{E\backslash A}$$


En cualquier espacio métrico $E$, todo subconjunto finito de $E$ es cerrado. 
$$ S(x,r) = \{ y\in E : d(y,x)=r \} = \overline{B}(x,r) \backslash B(x,r) $$
$$ Fr(A) = \overline{A}\backslash A^o = \overline{A}\cap (E\backslash A^o) \hspace{2cm}
Fr(A) = Fr(E\backslash A)  $$
$$ A \text{ es abierto } \Longleftrightarrow A\cap Fr(A) =0 \hspace{2cm}
A \text{ es cerrado }\Longleftrightarrow Fr(A)\subset A $$

Cuando todos los puntos de $A$ son aislados es un subconjunto discreto, es decir $A\cap A'=0$. Equivale a que la topología inducida sea la discreta.
En todo espacio métrico un punto es adherente a un conjunto si, y sólo si, existe una sucesión de puntos del conjunto que converge a él. 
$$ x\in A' \Longleftrightarrow 
U\cap (A\backslash\{x\}) \not=0 \hspace{0.5cm} \forall U\in\mathcal{U}(x)\Longleftrightarrow
B(x,\epsilon) \cap (A\backslash\{x\}) \not =0\hspace{0.5cm} \forall\epsilon\in\R^+ $$
$$ \{x_n\}\rightarrow x \Longleftrightarrow [\forall U\in\mathcal{U}(x) \ \exists m\in\N : n\geq m\implies x_n\in U] \Longleftrightarrow \{d(x_n,x)\}\rightarrow 0$$

Sean $d_1,d_2$ distancias, equivalen
\begin{itemize}
	\item La topología generada por $d_1$ está incluida en la generada por $d_2$
	\item Toda sucesión convergente para la distancia $d_2$ es convergente para $d_1$
\end{itemize}
Por tanto las distancias son equivalentes si, y sólo si, dan lugar a las mismas sucesiones convergentes.




Para convertir un producto de espacios normados en normado podemos tomar la norma del máximo.

\section{Continuidad y límite funcional}
Una función $f:E\rightarrow\R$ es continua en $x\in E$
$$ \forall V\in\mathcal{U}(f(x)) \ \exists U\in \mathcal{U}(x) : f(U) \subset V\hspace{2cm}
\forall\epsilon>0 \ \exists\delta>0 : y\in E,\ d(y,x)<\delta \implies d(f(x),f(y))<\epsilon$$
$$V\in\mathcal{U}(f(x)) \implies f^{-1}(V)\in\mathcal{U}(x)  \hspace{2cm}
x_n\in E\ \forall n\in\N,\ \{x_n\}\rightarrow x\implies \{f(x_n)\}\rightarrow f(x) $$

%%%%%%%%%%%%%%%%%%%%%%%%%%%%%%%%%%%%%%%%%%%%%%%%%%%%%%%%%%%%%%%%%%%%%%%%%%%%%%%%%%

% % % % % % % % % % % % % % % % % % % % % % % % % % % % % % % % %
%					 Bibliografía
% % % % % % % % % % % % % % % % % % % % % % % % % % % % % % % % %

\end{document}
