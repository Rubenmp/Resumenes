\documentclass[11pt,spanish]{article} % Idioma
\usepackage{babel}
\usepackage[T1]{fontenc}
\usepackage{textcomp, verbatim} % \begin{comment}
\usepackage[utf8]{inputenc} % Permite acentos

\usepackage{wrapfig} % Imagenes %\graphicspath{ {./imagenes/} }
\usepackage[left=2.75cm,top=2.5cm,right=2cm,bottom=2.5cm]{geometry} % Márgenes
\usepackage{amssymb, amsmath, amscd, amsfonts, amsthm, mathrsfs } % Símbolos matemáticos
\usepackage{cancel} % Cancelar expresiones
\usepackage{multirow, multicol, tabularx, booktabs, longtable} % Tablas
\usepackage{fancyhdr, fncychap} % Encabezados
\usepackage{algpseudocode, algorithmicx, algorithm} % Pseudo-código	
\usepackage{bbding} % Símbolos
\usepackage{enumitem} % Enumerados a), b), c)... usando \begin{enumerate}[label=\alph*)]
\usepackage{graphicx, xcolor, color, pstricks} % Gráficos --TikZ-- 
% http://www.texample.net/tikz/examples/
\usepackage[hidelinks]{hyperref}  % Enlaces
\usepackage{verbatim} % Comentarios largos \begin{comment}
\usepackage{rotating} % \begin{rotate}{30}
\usepackage[all]{xy} % Diagramas
\usepackage{listings} % Escribir código 
\usepackage{xparse} % Entornos


% Comandos
\newcommand{\docdate}{}
\newcommand{\subject}{}
\newcommand{\docauthor}{Rubén Morales Pérez}
\newcommand{\docemail}{srmorales@correo.ugr.es}

\newcommand{\N}{\mathbb{N}}
\newcommand{\Q}{\mathbb{Q}}
\newcommand{\C}{\mathbb{C}}
\newcommand{\R}{\mathbb{R}}
\newcommand{\Z}{\mathbb{Z}}


\linespread{1.1}                  % Espacio entre líneas.
\setlength\parindent{0pt}         % Indentación para párrafo.

\title{Análisis I}
\author{ }
\date{ }

% % % % % % % % % % % % % % % % % % % % % % % % % % % % % % % % %
%					 Inicio del documento
% % % % % % % % % % % % % % % % % % % % % % % % % % % % % % % % %
\begin{document}

\maketitle
\setlength\parindent{0pt} % Quitamos la sangría


\section{El espacio euclídeo, espacios normados y espacios métricos}

$\R^n = \R \text{x}\R \text{x }\overset{(N)}{\cdots} \text{ x}\R$ con la suma y producto por escalares tiene estructura de espacio vectorial. La base usual es $\Phi =\{e_k : k\in I_N\}$ con $e_k(k)=1$ y $e_k(j) = 0 \ \forall j\in I_N\backslash \{k\}$.

El producto escalar de dos vectores $x,y\in \R^N$ es $(x|y) = \sum_{k=1}^{N} x(k) y(k)$ cumple:
\begin{itemize}
	\item \textbf{(P.1)} $(\lambda u+\mu v) = \lambda (u|y) +\mu (v|y) \hspace{0.5cm} \forall u,v,y\in\R^N, \ \forall \lambda,\mu\in\R$
	\item \textbf{(P.2)} $(x|y) = (y|x) \hspace{0.5cm} \forall x,y\in\R^N$
	\item \textbf{(P.3)} $(x|x)>0 \hspace{0.5cm} \forall x\in\R^N\backslash\{0\}$
\end{itemize}
$\phi$ es una forma bilineal cuando es lineal en cada variable, simétrica si $\phi(x,y) = \phi(y,x)$. La forma cuadrática asociada se define como $Q(x) = \phi(x,x)$.
Un espacio pre-hilbertiano es un espacio vectorial con un producto escalar. 

La norma de un vector $x\in X$ es $||x|| = (x|x)^{1/2}$, intuitivamente es la longitud del vector. Cumple:
\begin{itemize}
	\item \textbf{(N.1)} $ ||x+y|| \leq ||x||+||y||\hspace{0.5cm} \forall x,y\in X \hspace{2cm} \text{(desigualdad triangular)}$
	\item \textbf{(N.2)} $||\lambda x|| = |\lambda| \ ||x|| \hspace{0.5cm} \forall x\in X,\ \forall\lambda\in\R \hspace{2cm}(homogeneidad por homotecias)$
	\item \textbf{(N.3)} $x\in X, \ ||x||=0 \implies x=0 \hspace{3cm} \text{(no degeneración)}$
\end{itemize}
\textbf{Desigualdad de Cauchy-Schwartz.} En todo espacio pre-hilbertiano X
\begin{center}
$ |(x|y)| \leq ||x|| \ ||y||\hspace{0.5cm}\forall x,y\in X
\hspace{2cm} \text{Igualdad sii $x,y$ linealmente dependientes}  $
\end{center}

Un espacio normado es un espacio vectorial con una norma $||\cdot||$. Todo espacio pre-hilbertiano es normado. 
Una norma puede no proceder de un producto escalar, ejemplo de ello son las normas del máximo y de la suma, con ellas $\R^N$ es espacio normado, no pre-hilbertiano.
$$ ||x||_{\infty} =\max \{|x_k| : k\in \{1,2,\cdots,N\} \hspace{2cm} ||x||_1 = \sum_{k=1}^{N} |x_k| $$

\subsection{Ortogonalidad/perpendicularidad}
$$ x\perp y \overset{\vartriangle}{\Longleftrightarrow} 
(x|y) = 0 \Longleftrightarrow
|| x+y ||^2 = ||x||^2 + ||y||^2$$
Un conjunto $A$ es ortogonal si $\forall x,y\in A, \ x\not =y, x\perp y$, se llama ortonormal si además $||x||=1 \ \forall x\in A$
$$ \text{Ángulo entre vectores no nulos } \alpha (x,y)=\arccos \frac{(x|y)}{||x||\ ||y||}\in[0,\pi]  $$

En un espacio normado se define la distancia por $d(x,y) = ||y-x||$. En cualquier conjunto definimos la distancia discreta como 
$$ \sigma (x,y) = 
\left\{ 
\begin{array}{l}
	0 \text{ si } x=y \\
	1 \text{ si } x\not =y \\
\end{array} \right.$$
\section{Topología de un espacio métrico}
Bola abierta $B(x,r) = \{y\in E : d(x,y)<r \} = \{x+ru : u\in B(0,1) \}$, su aspecto cambio cuando utilizamos una norma distinta de la euclídea.
Los abiertos de un espacio métrico son las uniones de bolas abiertas.
Dos distancias son equivalentes cuando generan la misma topología, dos normas lo son si lo son sus distancias asociadas.
Para dos normas $||\cdot||_1$ y $||\cdot||_2$ en un espacio vectorial equivalen:
\begin{itemize}
	\item $\exists p\in\R^+ : ||x||_2\leq p||x||_1 \ \forall x$
	\item La topología de la norma $||x||_2$ está incluida en la de $||x||_1$
\end{itemize}
Todas las normas en $\R^N$ son equivalentes.
Si $\tau$ es una topoloía en $E$ y $\tau_A$ la de un subespacio métrico
$$ \tau_A = \{ U\cap A : U\in\tau \} \hspace{2cm}
A^o = \bigcup \{U\in\tau : U\subset A \hspace{2cm}
\overline{A} = \bigcap \{ C\in\mathcal{C} : A\in\mathcal{C} \}$$
EL interior es el máximo abierto incluido en $A$, un conjunto es abierto si, y sólo si, es entorno de todos sus puntos. Un conjunto es denso en $\R$ si su adherencia es $\R$
$$ x\in A^o \Longleftrightarrow A\in\mathcal{U}(x)
\Longleftrightarrow \exists\epsilon>0 : B(x,\epsilon)\subset A $$
$$ x\in\overline{A}\text{ punto adherente } \Longleftrightarrow
U\cap A \not =0 \hspace{0.5cm} \forall U\in\mathcal{U}(x)\Longleftrightarrow
B(x,\epsilon)\cap A \not=0 \hspace{0.5cm} \forall\epsilon\in\R^+ $$
$$ E\backslash \overline{A} = (E\backslash A)^o \hspace{2cm}
 E\backslash A^o = \overline{E\backslash A}$$


En cualquier espacio métrico $E$, todo subconjunto finito de $E$ es cerrado. 
$$ S(x,r) = \{ y\in E : d(y,x)=r \} = \overline{B}(x,r) \backslash B(x,r) $$
$$ Fr(A) = \overline{A}\backslash A^o = \overline{A}\cap (E\backslash A^o) \hspace{2cm}
Fr(A) = Fr(E\backslash A)  $$
$$ A \text{ es abierto } \Longleftrightarrow A\cap Fr(A) =0 \hspace{2cm}
A \text{ es cerrado }\Longleftrightarrow Fr(A)\subset A $$

Cuando todos los puntos de $A$ son aislados es un subconjunto discreto, es decir $A\cap A'=0$. Equivale a que la topología inducida sea la discreta.
En todo espacio métrico un punto es adherente a un conjunto si, y sólo si, existe una sucesión de puntos del conjunto que converge a él. 
$$ x\in A' \Longleftrightarrow 
U\cap (A\backslash\{x\}) \not=0 \hspace{0.5cm} \forall U\in\mathcal{U}(x)\Longleftrightarrow
B(x,\epsilon) \cap (A\backslash\{x\}) \not =0\hspace{0.5cm} \forall\epsilon\in\R^+ $$
$$ \{x_n\}\rightarrow x \Longleftrightarrow [\forall U\in\mathcal{U}(x) \ \exists m\in\N : n\geq m\implies x_n\in U] \Longleftrightarrow \{d(x_n,x)\}\rightarrow 0$$

Sean $d_1,d_2$ distancias, equivalen
\begin{itemize}
	\item La topología generada por $d_1$ está incluida en la generada por $d_2$
	\item Toda sucesión convergente para la distancia $d_2$ es convergente para $d_1$
\end{itemize}
Por tanto las distancias son equivalentes si, y sólo si, dan lugar a las mismas sucesiones convergentes.




Para convertir un producto de espacios normados en normado podemos tomar la norma del máximo.

\section{Continuidad y límite funcional}
Una función $f:E\rightarrow\R$ es continua en $x\in E$
$$ \forall V\in\mathcal{U}(f(x)) \ \exists U\in \mathcal{U}(x) : f(U) \subset V\hspace{2cm}
\forall\epsilon>0 \ \exists\delta>0 : y\in E,\ d(y,x)<\delta \implies d(f(x),f(y))<\epsilon$$
$$V\in\mathcal{U}(f(x)) \implies f^{-1}(V)\in\mathcal{U}(x)  \hspace{2cm}
x_n\in E\ \forall n\in\N,\ \{x_n\}\rightarrow x\implies \{f(x_n)\}\rightarrow f(x) $$
\section{Compacidad y conexión}
La acotación no es una propiedad topológica, se conserva en un espacio normado ante normas equivalentes. 
$A$ está acotado si, y sólo si, el conjunto $\{|y-x| : x,y\in A\}$ está mayorado, si, y sólo si, está contenido en una bola.
Definimos el diámetro de un conjunto $diam\ A=\sup \{d(x,y) : x,y\in A\}=\sup \ A-\inf \ A$, $diam \ A=0\Longleftrightarrow A=\emptyset \text{ ó } A=\{x\}$

En cualquier espacio métrico toda sucesión convergente está acotada. Dos distancias que dan lugar a las mismas sucesiones convergentes son equivalentes.
$$ \forall x,y\in E \hspace{0.5cm}p(x,y) = \frac{d(x,y)}{1+d(x,y)}\text{ es distancia equivalente a d, con distintos conjuntos acotados} $$
En un espacio normado todas las normas son equivalentes.
Dos normas equivalentes en un espacio vectorial dan lugar a los mismos subconjuntos acotados.

En un producto de espacios normados un conjunto está acotado si, y sólo si, la restricción a cada espacio normado está acotada. Equivalente para una sucesión acotada.

\textbf{Teorema de Bolzano-Weierstrass. (pag 49)} Toda sucesión acoada de vectores de $\R^N$ admite una sucesión parcial convergente.
	
En todo espacio normado de dimensión infinita existe una sucesión acotada que no admite ninguna sucesión parcial convergente.

En espacios métricos nuestro concepto de compacto (secuencialmente compacto) nos dice que toda sucesión de puntos del mismo admite una parcial convergente.
Sea $E$ un espacio métrico y $A\subset E$:
\begin{itemize}
	\item Si $A$ es compacto, entonces $A$ es cerrado en $E$
	\item Si $E$ es compacto y $A$ cerrado, entonces $A$ es compacto
	\item Todo espacio métrico compacto está acotado
	\item Un subconjunto de $\R^N$ es compacto si, y sólo si, es cerrado y acotado (conexo en $\R$ $\Longleftrightarrow$ intervalo)
\end{itemize}
\textbf{Teorema de Hausdorff.} Todas las normas en $\R^N$ son equivalentes

Un espacio métrico es conexo cuando no se puede expresar como unión de dos subconjuntos abiertos, no vacíos y disjuntos.

Para un espacio métrico $E$ son equivalentes
\begin{itemize}
	\item $E$ es conexo
	\item La imagen de toda función continua de $E$ en $\R$ es un intervalo
	\item Toda función continua de $E$ en $\{0,1\}$ es constante
\end{itemize}
Un subconjunto $E$ de un espacio vectorial $X$ es convexo cuando, para cualesquiera dos puntos de $E$, el segmento que los une está contenido en $E$, es decir, $ x,y\in E\implies \{ (1-t)x+ty : t\in[0,1] \} \subset E$

Todo subconjunto convexo de un espacio normado es conexo.

$E$ es conexo por arcos cuando, para cualesquiera $x,y\in E$ existe una función continua $f:[0,1]\rightarrow E$, tal que $f(0) = x$ y $f(1) = y$.
\begin{center}
	convexo $\implies$ conexo por arcos $\implies$ conexo
\end{center}
\section{Complitud y continuidad uniforme}
$\{x_n\}$ es una sucesión de Cauchy en un espacio métrico cuando
$$ \forall\epsilon>0, \ \exists m\in\N : p,q\geq m\implies d(x_p,x_q) \geq\epsilon$$
En cualquier espacio métrico, toda sucesión convergente es una sucesión de Cauchy (recíproco falso). Tomando una distancia $p(x,y) = |e^y-e^y| \ \forall x,y\in\R$ y la sucesión $\{-n\}$ para la distancia $p$ es sucesión de cauchy pero no lo es para la distancia usual.

Una distancia es completa cuando toda sucesión de Cauchy es convergente. La complitud no es una propiedad topológica, puede variar al cambiar la distancia.

Un espacio normado cuya norma es completa (distancia asociada completa) se llama espacio de Banach.

Un espacio pre-hilbertiano cuya norma (asociada al producto escalar) es completa se llama espacio de Hilbert.

Dos normas equivalentes en un mismo espacio vectorial dan lugar a las mismas sucesiones de Cauchy.

\textbf{Teorema.} Todo espacio normado de dimensión finita es un espacio de Banach. Por tanto, el espacio euclídeo N-dimensional es un espacio de Hilbert.

Sea $E$ un espacio métrico y $A$ un subespacio métrico de $E$:
\begin{itemize}
	\item Si $A$ es completo, entonces $A$ es un subconjunto cerrado de $E$
	\item Si $E$ es completo y $A$ es un subconjunto cerrado de $E$, entonces $A$ es completo
\end{itemize}
Si $X$ es un espacio normado arbitrario, todo subespacio de dimensión finita de $X$ es un subconjunto cerrado de $X$.
Decimos que $f$ es uniformemente continua cuando
$$ \forall\epsilon>0,\ \exists\delta>0 : x,y\in E, \ d(x,y)<\delta\implies d(f(x),f(y))<\epsilon $$
Si $f$ es uniformemente continua y $\{x_n\},\{y_n\}$ dos sucesiones de puntos de $E$ con $\{d(x_n,y_n)\} \rightarrow 0$, entonces $\{d(f(x), f(y))\} \rightarrow 0$. Si $f$ no es uniformemente continua existen $\{x_n\},\{y_n\}$ de puntos de $E$ y existe $\epsilon>0 : d(x_n,y_n)<1/n\ \forall n\in\N$, pero $d(f(x_n), f(y_n))\geq\epsilon\ \forall n\in\N$
La continuidad uniforme no es una propiedad local.

\textbf{Teorema de Heine (pag 63)} Sean $E$ y $F$ espacios métricos y $f:E\rightarrow F$ una función continua. Si $E$ es compacto, entonces $f$ es uniformemente continua.

Sean $E,F$ espacios métricos y $f:E\rightarrow F$ uniformemente continua. Si $\{x_n\}$ es sucesión de Cauchy en $E$, $\{f(x_n)\}$ es sucesión de Cauchy en $F$

Sean $E,F$ espacios métricos, $F$ completo, $\emptyset \not =A\subset E$ y $f:A\rightarrow F$ uniformemente continua. Entonces existe una única función continua $g:\overline{A}\rightarrow F$ uniformemente continua que extiende a $f$

Una función uniformemente continua puede dejar de serlo al cambiar la distancia de destino por otra equivalente, pero se mantiene con normas equivalentes.

\subsection{Funciones lipschitzianas}
Una función $f$ entre espacios métricos es lipschitziana cuando existe una constante $M\in\R^+_0$ tal que 
$$ d(f(x), f(y)) \leq Md(x,y) \hspace{0.5cm} \forall x,y\in E\hspace{2cm}\text{lipschitziana$\implies$uniformemente continua}$$
Si la constante de Lipschitz $M_0\leq 1$ se dice que $f$ es no expansiva, si es $M_0<1$ es contractiva
\subsection{Teorema del punto fijo de Banach (pag 66)}
Sea $E$ un espacio métrico completo y $f:E\rightarrow E$ contractiva. Entones $\exists !\ x\in E : f(x) = x$

\textbf{Teorema}
Sean $X,Y$ espacios normados y $T:X\rightarrow Y$ una aplicación lineal. Son equivalentes:
\begin{itemize}
	\item Existe $x_0\in X$ tal que $T$ es continua en $x_0$
	\item $T$ es continua en $0$
	\item $T$ es continua
	\item $T$ es uniformemente continua
	\item $T$ es lipschitziana
	\item Existe $m\in\R^+_0$ tal que $||T(x)||\leq M||x||\hspace{0.5cm} \forall x\in X$
	\item Si $A$ es un subconjunto acotado de $X$, entonces $T(A)$ es un subconjunto acotado de $Y$
	\item $T$ está acotada en la bola cerrada unidad de $X$
	\item $T$ está acotada en la esfera unidad de $X$
\end{itemize}
Si $X$ es un espacio normado de dimensión finita, toda aplicación lineal de $X$ en cualquier otro espacio normado.

La norma de una aplicación lineal continua $T$, $||T||$, es la constante de Lipschitz de $T$
\section{Diferenciabilidad}
Sean $X,Y$ espacios normados y $f:A\subset X\rightarrow Y$, $a\in\overset{o}{A}$ $f$ es diferenciable en $a$ cuando existe $T\in L(X,Y)$ continua verificando
$$ \lim_{x\rightarrow a} \frac{||f(x)-f(a)-T(x-a)||}{||x-a||}$$
Si solamente tenemos $a\in A\cap \overline{A}$ no podemos asegurar la unicidad de la diferencial, con $a\in\overset{o}{A}$ si.

Toda aplicación lineal cuando $X$ tiene dimensión finita es continua.
La diferenciabilidad tiene carácter local y se conserva al sustituir las normas por otras equivalentes.
La diferenciación es una operación lineal.

\textbf{Regla de la cadena (pag 79).} Sean $X,Y,Z$ espacios normados, $\Omega\subset X, U\subset Y$ abiertos y $f:\Omega\rightarrow U$, $g:U\rightarrow Z$ dos funciones. Si $f$ es diferenciable en $a\in\Omega$ y $g$ es diferenciable en $f(a)$, entonces $gof$ es diferenciable en $a$, con $D(gof)(a) = Dg(f(a))oDf(a)$

$f$ es diferenciable en un punto equivale a que todas sus componentes sean diferenciables.
\section{Vector derivada}
Sea $Y$ espacio normado y $\Omega\subset\R$ abierto. Una función $f:\Omega\rightarrow Y$ es diferenciable en $a\in\Omega$ si, y sólo si, $f$ es derivable en $a$, la diferencial y el vector derivada quedan determinados por
$$ f'(a) = Df(a)(1)\hspace{2cm} Df(a)(t) = tf'(a)\hspace{0.5cm} \forall t\in\R $$
Si $f$ es derivable en $a\in\Omega$, entonces $\forall\epsilon>0,\ \exists\delta>0$ tal que
$$ \left.
	\begin{array}{l}
		t_1,t_2\in J,\ t_1\not =t_2 \\
		a-\delta<t_1\leq a\leq t_2<a+\delta \\
	\end{array} \right\}
\implies
\left\|
\frac{f(t_2)-f(t_1)}{t_2-t_1} -f'(a)\right\| \leq\epsilon
$$
$f=(f_1,f_2,\cdots,f_M)$ es derivable si, y sólo si lo es $\forall f_i$, entonces
$$ f_k'(a) = \pi_k(f'(a))\hspace{0.5cm}\forall k\in I_M\hspace{2cm} 
f'(a) = \sum_{k=1}^{M} f_k'(a)\cdot e_k$$

Dada una función $\phi:A\subset\R\rightarrow\R^M$, $\phi (J)$ es una curva paramétrica en $\R^M$. 
La recta tangente a la curva en el punto $\phi (a)$ es $R=\{ \phi(a)+t\phi'(a) : t\in\R \}$.
Cuando $\phi'(a)\not=0$ $\phi(a)$ es un punto regular de la curva, en otro caso es un punto singular.
Cuando la ordenada va en función de la abscisa se denomina curva explícita.
Toda curva explícita puede verse como una paramétrica, el recíproco es falso.
\section{Vector gradiente}
En este caso estudiamos funciones entre espacios normados $f:\R^N\rightarrow\R$ con $N>1$.

Si $u\in S=\{ u\in\R^N : ||u||=1 \}$, $f$ es derivable en la dirección $u$, en el punto $a$ cuando $\sigma_u:]-r,r[\rightarrow Y, \sigma_u(t)=f(a+tu)\ \forall t\in]-r,r[$, es derivable en $0$
$$ f_u'(a) = \sigma_u'(0) = \lim_{t\rightarrow 0} \frac{f(a+tu)-f(a)}{t}\in Y $$

Sean $X,Y$ espacios normados, $\empty\not=\Omega\subset X$ abierto, $f:\Omega\rightarrow Y$. Si $f$ es diferenciable en $a\in\Omega$, entonces $f$ es direccionalmente derivable en $a$, y para todo $u\in X$ con $||u||=1$, se tiene $f_u'(a) = Df(a) (u)$

Definimos las derivadas parciales
$$ \frac{\partial f}{\partial x_j} (a) = f_{e_j}'(a) = \lim_{t\rightarrow 0} \frac{f(a+te_j)-f(a)}{t} $$
Si esto ocurre en todas las direcciones decimos que $f$ es parcialmente derivable.

Si $f$ es diferenciable en $a$, entonces $f$ es parcialmente derivable en $a$ con
$$ \frac{\partial f}{\partial x_j} (a) = Df(a)(e_j) \hspace{0.5cm}\forall j\in I_N\hspace{2cm}
\frac{\partial f}{\partial x} (x_0,y_0) = \lim_{x\rightarrow x_o} \frac{f(x,y_o)-f(x_0,y_0)}{x-x_0} $$
Para calcular la derivada $x_j$ es la única variable, y tratar todas las demás como constantes.

Si $f$ es diferenciable en $a$ para todo $x\in\R^N$
$$ Df(a)(x) = \sum_{k=1}^{N} x_j\frac{\partial f}{\partial x_j} (a)\hspace{1.5cm}
\nabla f(a) = \left( \frac{\partial f}{\partial x_1}(a), \frac{\partial f}{\partial x_2}(a),\cdots,\frac{\partial f}{\partial x_N}(a),   \right) = \sum_{j=1}^{N}\frac{\partial f}{\partial x_j} (a)e_j $$
Son equivalentes
\begin{itemize}
	\item $f$ es diferenciable en $a$
	\item $f$ es parcialmente derivable en $a$ y se verifica que
	$$ \lim_{x\rightarrow a} \frac{f(x)-f(a)-(\nabla f(a) | x-a)}{||x-a||} = 0$$
\end{itemize}
en caso de que se cumplan ambas se tiene
$$ Df(a)(x) = ( \nabla f(a) | x ) \hspace{0.5cm} \forall x\in\R^N$$
Sea $\phi : L(\R^N,\R) \rightarrow \R^N$ la aplicación definida por $\phi(t) = \sum_{j=1}^{N} T(e_j) e_j$ para cada $T\in L(\R^N,\R)$. Se tiene que $\phi$ es lineal, biyectiva y conserva la norma, luego permite identificar totalmente los espacios normados $L(\R^N,\R)$ y $\R^N$

Si el campo escalar $f$ es diferenciable en $a\in\Omega$, sus derivadas direccionales en el punto $a$ vienen dadas por
$$ f_u'(a) = Df(a)(u) = (\nabla f(a) | u) \hspace{0.5cm} \forall u\in S\hspace{2cm}
f_v'(a)$$
Si $\nabla f(a)=0$ entonces $a$  es un punto crítico o estacionario del campo $f$.

\subsection{Plano tangente a superficie explícita}
Sea $S=Gr\ f = \{ (x,y,f(x,y)) : (x,y)\in\Omega \} \subset \R^3$ y $f:\Omega\rightarrow\R$ diferenciable en $(x_0,y_0)$ el plano $\Pi$ de ecuación
$$ \frac{\partial f}{\partial x} (x_0,y_0)(x-x_0) + \frac{\partial f}{\partial y}(x_0,y_0)(y-y_0)-(z-z_0) = 0 $$
es el plano tangente a la superficie $S$ en el punto $p=(x_0,y_0,z_0)$ y el vector normal que pasa por $p$ es
$$ \left( \frac{\partial f}{\partial x} (x_0,y_0), \frac{\partial f}{\partial y}(x_0,y_0), -1 \right) \in\R^3$$
Sin embargo solamente es una buena aproximación cuando $f$ es diferenciable en $(x_0,y_0)$
\section{Matriz jacobiana}
Nos encargamos ahora de las aplicaciones $f:\R^N\rightarrow\R^M$ con $N,M>1$. Para todo $x\in\R^N\ \exists! Jf(a)\in\mathcal{M}_{M\text{x}N} : y=T(x)\in\R^M,y=Jf(a)x$. La matriz de la aplicación es única.
$$ Jf(a) = \alpha_{kj}= Df(a)(e_j) = \frac{\partial f_k}{\partial x_j}(a) \hspace{0.5cm} \forall k\in I_M,\ \forall j\in I_N $$
$$ Jf(a) = \left(
\begin{matrix}
	\frac{\partial f_1}{\partial x_1} (a) & 	\frac{\partial f_1}{\partial x_2} (a) & \cdots & \cdots 	\frac{\partial f_1}{\partial x_N} (a) \\
	\frac{\partial f_2}{\partial x_1} (a) & 	\frac{\partial f_2}{\partial x_2} (a) & \cdots & \cdots 	\frac{\partial f_2}{\partial x_N} (a) \\	
	\vdots & \vdots & & & \vdots \\
	\frac{\partial f_M}{\partial x_1} (a) & 	\frac{\partial f_M}{\partial x_2} (a) & \cdots & \cdots 	\frac{\partial f_M}{\partial x_N} (a) \\	
\end{matrix}
\right) $$
Cada columna de la matriz puede entenderse como una derivada parcial.

Sea $U$ abierto de $\R^M$ tal que $f(\Omega) \subset U$ y $G:U\rightarrow \R^P, \ P\in\N$. Tomamos la composición $h=gof:\Omega\rightarrow\R^P$, suponemos $f$ diferenciable en $a$ y $g$ diferenciable en $f(a)$, por tanto
$$ Dh(a) = Dg(f(a)) o Df(a) \hspace{2cm}
z = (Jg(b)\cdot Jf(a))\cdot x\hspace{0.5cm}\forall x\in\R^N$$

\subsection{Coordenadas polares en el plano}
$M=N=2, \Omega=\R^+\text{x}]0,\pi[$, y $f:\Omega\rightarrow\R^2$ con componentes $x,y$. $f(\Omega)$ es el semiplano superior.
$$ x(p,\theta) = p\cos \theta\hspace{1.5cm} y(p,\theta) = p\sin\theta\hspace{0.5cm} \forall (p,\theta)\in\Omega$$

%%%%%%%%%%%%%%%%%%%%%%%%%%%%%%%%%%%%%%%%%%%%%%%%%%%%%%%%%%%%%%%%%%%%%%%%%%%%%%%%%%

% % % % % % % % % % % % % % % % % % % % % % % % % % % % % % % % %
%					 Bibliografía
% % % % % % % % % % % % % % % % % % % % % % % % % % % % % % % % %

\end{document}
