\section{Vector derivada}
Sea $Y$ espacio normado y $\Omega\subset\R$ abierto. Una función $f:\Omega\rightarrow Y$ es diferenciable en $a\in\Omega$ si, y sólo si, $f$ es derivable en $a$, la diferencial y el vector derivada quedan determinados por
$$ f'(a) = Df(a)(1)\hspace{2cm} Df(a)(t) = tf'(a)\hspace{0.5cm} \forall t\in\R $$
Si $f$ es derivable en $a\in\Omega$, entonces $\forall\epsilon>0,\ \exists\delta>0$ tal que
$$ \left.
	\begin{array}{l}
		t_1,t_2\in J,\ t_1\not =t_2 \\
		a-\delta<t_1\leq a\leq t_2<a+\delta \\
	\end{array} \right\}
\implies
\left\|
\frac{f(t_2)-f(t_1)}{t_2-t_1} -f'(a)\right\| \leq\epsilon
$$
$f=(f_1,f_2,\cdots,f_M)$ es derivable si, y sólo si lo es $\forall f_i$, entonces
$$ f_k'(a) = \pi_k(f'(a))\hspace{0.5cm}\forall k\in I_M\hspace{2cm} 
f'(a) = \sum_{k=1}^{M} f_k'(a)\cdot e_k$$

Dada una función $\phi:A\subset\R\rightarrow\R^M$, $\phi (J)$ es una curva paramétrica en $\R^M$. 
La recta tangente a la curva en el punto $\phi (a)$ es $R=\{ \phi(a)+t\phi'(a) : t\in\R \}$.
Cuando $\phi'(a)\not=0$ $\phi(a)$ es un punto regular de la curva, en otro caso es un punto singular.
Cuando la ordenada va en función de la abscisa se denomina curva explícita.
Toda curva explícita puede verse como una paramétrica, el recíproco es falso.