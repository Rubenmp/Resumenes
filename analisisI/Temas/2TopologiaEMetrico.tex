\section{Topología de un espacio métrico}
Bola abierta $B(x,r) = \{y\in E : d(x,y)<r \} = \{x+ru : u\in B(0,1) \}$, su aspecto cambio cuando utilizamos una norma distinta de la euclídea.
Los abiertos de un espacio métrico son las uniones de bolas abiertas.
Dos distancias son equivalentes cuando generan la misma topología, dos normas lo son si lo son sus distancias asociadas.
Para dos normas $||\cdot||_1$ y $||\cdot||_2$ en un espacio vectorial equivalen:
\begin{itemize}
	\item $\exists p\in\R^+ : ||x||_2\leq p||x||_1 \ \forall x$
	\item La topología de la norma $||x||_2$ está incluida en la de $||x||_1$
\end{itemize}
Todas las normas en $\R^N$ son equivalentes.
Si $\tau$ es una topoloía en $E$ y $\tau_A$ la de un subespacio métrico
$$ \tau_A = \{ U\cap A : U\in\tau \} \hspace{2cm}
A^o = \bigcup \{U\in\tau : U\subset A \hspace{2cm}
\overline{A} = \bigcap \{ C\in\mathcal{C} : A\in\mathcal{C} \}$$
EL interior es el máximo abierto incluido en $A$, un conjunto es abierto si, y sólo si, es entorno de todos sus puntos. Un conjunto es denso en $\R$ si su adherencia es $\R$
$$ x\in A^o \Longleftrightarrow A\in\mathcal{U}(x)
\Longleftrightarrow \exists\epsilon>0 : B(x,\epsilon)\subset A $$
$$ x\in\overline{A}\text{ punto adherente } \Longleftrightarrow
U\cap A \not =0 \hspace{0.5cm} \forall U\in\mathcal{U}(x)\Longleftrightarrow
B(x,\epsilon)\cap A \not=0 \hspace{0.5cm} \forall\epsilon\in\R^+ $$
$$ E\backslash \overline{A} = (E\backslash A)^o \hspace{2cm}
 E\backslash A^o = \overline{E\backslash A}$$


En cualquier espacio métrico $E$, todo subconjunto finito de $E$ es cerrado. 
$$ S(x,r) = \{ y\in E : d(y,x)=r \} = \overline{B}(x,r) \backslash B(x,r) $$
$$ Fr(A) = \overline{A}\backslash A^o = \overline{A}\cap (E\backslash A^o) \hspace{2cm}
Fr(A) = Fr(E\backslash A)  $$
$$ A \text{ es abierto } \Longleftrightarrow A\cap Fr(A) =0 \hspace{2cm}
A \text{ es cerrado }\Longleftrightarrow Fr(A)\subset A $$

Cuando todos los puntos de $A$ son aislados es un subconjunto discreto, es decir $A\cap A'=0$. Equivale a que la topología inducida sea la discreta.
En todo espacio métrico un punto es adherente a un conjunto si, y sólo si, existe una sucesión de puntos del conjunto que converge a él. 
$$ x\in A' \Longleftrightarrow 
U\cap (A\backslash\{x\}) \not=0 \hspace{0.5cm} \forall U\in\mathcal{U}(x)\Longleftrightarrow
B(x,\epsilon) \cap (A\backslash\{x\}) \not =0\hspace{0.5cm} \forall\epsilon\in\R^+ $$
$$ \{x_n\}\rightarrow x \Longleftrightarrow [\forall U\in\mathcal{U}(x) \ \exists m\in\N : n\geq m\implies x_n\in U] \Longleftrightarrow \{d(x_n,x)\}\rightarrow 0$$

Sean $d_1,d_2$ distancias, equivalen
\begin{itemize}
	\item La topología generada por $d_1$ está incluida en la generada por $d_2$
	\item Toda sucesión convergente para la distancia $d_2$ es convergente para $d_1$
\end{itemize}
Por tanto las distancias son equivalentes si, y sólo si, dan lugar a las mismas sucesiones convergentes.




Para convertir un producto de espacios normados en normado podemos tomar la norma del máximo.
