\section{Diferenciabilidad}
Sean $X,Y$ espacios normados y $f:A\subset X\rightarrow Y$, $a\in\overset{o}{A}$ $f$ es diferenciable en $a$ cuando existe $T\in L(X,Y)$ continua verificando
$$ \lim_{x\rightarrow a} \frac{||f(x)-f(a)-T(x-a)||}{||x-a||}$$
Si solamente tenemos $a\in A\cap \overline{A}$ no podemos asegurar la unicidad de la diferencial, con $a\in\overset{o}{A}$ si.

Toda aplicación lineal cuando $X$ tiene dimensión finita es continua.
La diferenciabilidad tiene carácter local y se conserva al sustituir las normas por otras equivalentes.
La diferenciación es una operación lineal.

\textbf{Regla de la cadena (pag 79).} Sean $X,Y,Z$ espacios normados, $\Omega\subset X, U\subset Y$ abiertos y $f:\Omega\rightarrow U$, $g:U\rightarrow Z$ dos funciones. Si $f$ es diferenciable en $a\in\Omega$ y $g$ es diferenciable en $f(a)$, entonces $gof$ es diferenciable en $a$, con $D(gof)(a) = Dg(f(a))oDf(a)$

$f$ es diferenciable en un punto equivale a que todas sus componentes sean diferenciables.