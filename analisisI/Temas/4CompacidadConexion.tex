\section{Compacidad y conexión}
La acotación no es una propiedad topológica, se conserva en un espacio normado ante normas equivalentes. 
$A$ está acotado si, y sólo si, el conjunto $\{|y-x| : x,y\in A\}$ está mayorado, si, y sólo si, está contenido en una bola.
Definimos el diámetro de un conjunto $diam\ A=\sup \{d(x,y) : x,y\in A\}=\sup \ A-\inf \ A$, $diam \ A=0\Longleftrightarrow A=\emptyset \text{ ó } A=\{x\}$

En cualquier espacio métrico toda sucesión convergente está acotada. Dos distancias que dan lugar a las mismas sucesiones convergentes son equivalentes.
$$ \forall x,y\in E \hspace{0.5cm}p(x,y) = \frac{d(x,y)}{1+d(x,y)}\text{ es distancia equivalente a d, con distintos conjuntos acotados} $$
En un espacio normado todas las normas son equivalentes.
Dos normas equivalentes en un espacio vectorial dan lugar a los mismos subconjuntos acotados.

En un producto de espacios normados un conjunto está acotado si, y sólo si, la restricción a cada espacio normado está acotada. Equivalente para una sucesión acotada.

\textbf{Teorema de Bolzano-Weierstrass. (pag 49)} Toda sucesión acoada de vectores de $\R^N$ admite una sucesión parcial convergente.
	
En todo espacio normado de dimensión infinita existe una sucesión acotada que no admite ninguna sucesión parcial convergente.

En espacios métricos nuestro concepto de compacto (secuencialmente compacto) nos dice que toda sucesión de puntos del mismo admite una parcial convergente.
Sea $E$ un espacio métrico y $A\subset E$:
\begin{itemize}
	\item Si $A$ es compacto, entonces $A$ es cerrado en $E$
	\item Si $E$ es compacto y $A$ cerrado, entonces $A$ es compacto
	\item Todo espacio métrico compacto está acotado
	\item Un subconjunto de $\R^N$ es compacto si, y sólo si, es cerrado y acotado (conexo en $\R$ $\Longleftrightarrow$ intervalo)
\end{itemize}
\textbf{Teorema de Hausdorff.} Todas las normas en $\R^N$ son equivalentes

Un espacio métrico es conexo cuando no se puede expresar como unión de dos subconjuntos abiertos, no vacíos y disjuntos.

Para un espacio métrico $E$ son equivalentes
\begin{itemize}
	\item $E$ es conexo
	\item La imagen de toda función continua de $E$ en $\R$ es un intervalo
	\item Toda función continua de $E$ en $\{0,1\}$ es constante
\end{itemize}
Un subconjunto $E$ de un espacio vectorial $X$ es convexo cuando, para cualesquiera dos puntos de $E$, el segmento que los une está contenido en $E$, es decir, $ x,y\in E\implies \{ (1-t)x+ty : t\in[0,1] \} \subset E$

Todo subconjunto convexo de un espacio normado es conexo.

$E$ es conexo por arcos cuando, para cualesquiera $x,y\in E$ existe una función continua $f:[0,1]\rightarrow E$, tal que $f(0) = x$ y $f(1) = y$.
\begin{center}
	convexo $\implies$ conexo por arcos $\implies$ conexo
\end{center}