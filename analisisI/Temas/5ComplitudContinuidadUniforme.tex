\section{Complitud y continuidad uniforme}
$\{x_n\}$ es una sucesión de Cauchy en un espacio métrico cuando
$$ \forall\epsilon>0, \ \exists m\in\N : p,q\geq m\implies d(x_p,x_q) \geq\epsilon$$
En cualquier espacio métrico, toda sucesión convergente es una sucesión de Cauchy (recíproco falso). Tomando una distancia $p(x,y) = |e^y-e^y| \ \forall x,y\in\R$ y la sucesión $\{-n\}$ para la distancia $p$ es sucesión de cauchy pero no lo es para la distancia usual.

Una distancia es completa cuando toda sucesión de Cauchy es convergente. La complitud no es una propiedad topológica, puede variar al cambiar la distancia.

Un espacio normado cuya norma es completa (distancia asociada completa) se llama espacio de Banach.

Un espacio pre-hilbertiano cuya norma (asociada al producto escalar) es completa se llama espacio de Hilbert.

Dos normas equivalentes en un mismo espacio vectorial dan lugar a las mismas sucesiones de Cauchy.

\textbf{Teorema.} Todo espacio normado de dimensión finita es un espacio de Banach. Por tanto, el espacio euclídeo N-dimensional es un espacio de Hilbert.

Sea $E$ un espacio métrico y $A$ un subespacio métrico de $E$:
\begin{itemize}
	\item Si $A$ es completo, entonces $A$ es un subconjunto cerrado de $E$
	\item Si $E$ es completo y $A$ es un subconjunto cerrado de $E$, entonces $A$ es completo
\end{itemize}
Si $X$ es un espacio normado arbitrario, todo subespacio de dimensión finita de $X$ es un subconjunto cerrado de $X$.
Decimos que $f$ es uniformemente continua cuando
$$ \forall\epsilon>0,\ \exists\delta>0 : x,y\in E, \ d(x,y)<\delta\implies d(f(x),f(y))<\epsilon $$
Si $f$ es uniformemente continua y $\{x_n\},\{y_n\}$ dos sucesiones de puntos de $E$ con $\{d(x_n,y_n)\} \rightarrow 0$, entonces $\{d(f(x), f(y))\} \rightarrow 0$. Si $f$ no es uniformemente continua existen $\{x_n\},\{y_n\}$ de puntos de $E$ y existe $\epsilon>0 : d(x_n,y_n)<1/n\ \forall n\in\N$, pero $d(f(x_n), f(y_n))\geq\epsilon\ \forall n\in\N$
La continuidad uniforme no es una propiedad local.

\textbf{Teorema de Heine (pag 63)} Sean $E$ y $F$ espacios métricos y $f:E\rightarrow F$ una función continua. Si $E$ es compacto, entonces $f$ es uniformemente continua.

Sean $E,F$ espacios métricos y $f:E\rightarrow F$ uniformemente continua. Si $\{x_n\}$ es sucesión de Cauchy en $E$, $\{f(x_n)\}$ es sucesión de Cauchy en $F$

Sean $E,F$ espacios métricos, $F$ completo, $\emptyset \not =A\subset E$ y $f:A\rightarrow F$ uniformemente continua. Entonces existe una única función continua $g:\overline{A}\rightarrow F$ uniformemente continua que extiende a $f$

Una función uniformemente continua puede dejar de serlo al cambiar la distancia de destino por otra equivalente, pero se mantiene con normas equivalentes.

\subsection{Funciones lipschitzianas}
Una función $f$ entre espacios métricos es lipschitziana cuando existe una constante $M\in\R^+_0$ tal que 
$$ d(f(x), f(y)) \leq Md(x,y) \hspace{0.5cm} \forall x,y\in E\hspace{2cm}\text{lipschitziana$\implies$uniformemente continua}$$
Si la constante de Lipschitz $M_0\leq 1$ se dice que $f$ es no expansiva, si es $M_0<1$ es contractiva
\subsection{Teorema del punto fijo de Banach (pag 66)}
Sea $E$ un espacio métrico completo y $f:E\rightarrow E$ contractiva. Entones $\exists !\ x\in E : f(x) = x$

\textbf{Teorema}
Sean $X,Y$ espacios normados y $T:X\rightarrow Y$ una aplicación lineal. Son equivalentes:
\begin{itemize}
	\item Existe $x_0\in X$ tal que $T$ es continua en $x_0$
	\item $T$ es continua en $0$
	\item $T$ es continua
	\item $T$ es uniformemente continua
	\item $T$ es lipschitziana
	\item Existe $m\in\R^+_0$ tal que $||T(x)||\leq M||x||\hspace{0.5cm} \forall x\in X$
	\item Si $A$ es un subconjunto acotado de $X$, entonces $T(A)$ es un subconjunto acotado de $Y$
	\item $T$ está acotada en la bola cerrada unidad de $X$
	\item $T$ está acotada en la esfera unidad de $X$
\end{itemize}
Si $X$ es un espacio normado de dimensión finita, toda aplicación lineal de $X$ en cualquier otro espacio normado.

La norma de una aplicación lineal continua $T$, $||T||$, es la constante de Lipschitz de $T$