\documentclass[11pt,spanish]{article} % Idioma
\usepackage{babel}
\usepackage[T1]{fontenc}
\usepackage{textcomp, verbatim} % \begin{comment}
\usepackage[utf8]{inputenc} % Permite acentos

\usepackage{wrapfig} % Imagenes %\graphicspath{ {./imagenes/} }
\usepackage[left=2.75cm,top=2.5cm,right=2cm,bottom=2.5cm]{geometry} % Márgenes
\usepackage{amssymb, amsmath, amscd, amsfonts, amsthm, mathrsfs } % Símbolos matemáticos
\usepackage{cancel} % Cancelar expresiones
\usepackage{multirow, multicol, tabularx, booktabs, longtable} % Tablas
\usepackage{fancyhdr, fncychap} % Encabezados
\usepackage{algpseudocode, algorithmicx, algorithm} % Pseudo-código	
\usepackage{bbding} % Símbolos
\usepackage{enumitem} % Enumerados a), b), c)... usando \begin{enumerate}[label=\alph*)]
\usepackage{graphicx, xcolor, color, pstricks} % Gráficos --TikZ-- 
% http://www.texample.net/tikz/examples/
\usepackage[hidelinks]{hyperref}  % Enlaces
\usepackage{verbatim} % Comentarios largos \begin{comment}
\usepackage{rotating} % \begin{rotate}{30}
\usepackage[all]{xy} % Diagramas
\usepackage{listings} % Escribir código 
\usepackage{xparse} % Entornos


% Comandos
\newcommand{\docdate}{}
\newcommand{\subject}{}
\newcommand{\docauthor}{Rubén Morales Pérez}
\newcommand{\docemail}{srmorales@correo.ugr.es}

\newcommand{\N}{\mathbb{N}}
\newcommand{\Q}{\mathbb{Q}}
\newcommand{\C}{\mathbb{C}}
\newcommand{\R}{\mathbb{R}}
\newcommand{\Z}{\mathbb{Z}}

\newcommand{\limn}{\lim_{\substack{n\rightarrow\infty}}}
\newcommand{\limx}{\lim_{\substack{x\rightarrow\infty}}}

\linespread{1.1}                  % Espacio entre líneas.
\setlength\parindent{0pt}         % Indentación para párrafo.

\title{Cálculo II}
\author{ }
\date{ }

% % % % % % % % % % % % % % % % % % % % % % % % % % % % % % % % %
%					 Inicio del documento
% % % % % % % % % % % % % % % % % % % % % % % % % % % % % % % % %
\begin{document}

\maketitle
\setlength\parindent{0pt} % Quitamos la sangría

\section{Límite funcional}
$\alpha\in\R$ es punto de acumulación de $A$ si existe  $\{x_n\}\rightarrow\alpha$ de puntos de $A\backslash\{\alpha \}$. 
$A'$ es el conjunto de todos estos puntos.
Los puntos $a\in A\backslash A'$ son aislados, toda función es continua allí.
$$ \alpha\in A' \Longleftrightarrow
]\alpha-\delta, \alpha +\delta[\cap(A\backslash\{\alpha\})\not =\emptyset \ \ \forall\delta>0 $$

Son equivalentes:
\begin{itemize}
	\item $\lim_{x\rightarrow\alpha\in A'} f(x) = L\in\R$
	\item $ x_n\in A\backslash\{\alpha\}\ \ \forall n\in\N, \ \{x_n\}\rightarrow\alpha \implies \{f(x_n)\}\rightarrow L \text{ (puede verse solamente con sucesiones monótonas)}$
	\item $ \forall\epsilon>0, \ \exists\delta>0 \ : \ x\in A, \ 0<|x-\alpha|<\delta \implies |f(x)-L|<\epsilon $
\end{itemize}

Una función $f:A\rightarrow\R$ es continua en un punto $a\in A\cap A'$ si, y sólo si, $\lim_{x\rightarrow\alpha} f(x) = f(a)$

Si una función tiene límite en $a\in A\cap A'$ pero $\lim_{x\rightarrow\alpha} f(x) \not=f(a)$ $f$ tiene una discontinuidad evitable en el punto $a$. Son equivalentes:
\begin{itemize}
	\item $f$ tiene límite en el punto $\alpha\in A\backslash A'$
	\item Existe $f':A\cup\{\alpha\} \rightarrow\R$ continua en $\alpha$ con $f'(x)=f(x) \ \ \forall x\in A$
\end{itemize}

\subsection{Límites laterales}
El límite tiene carácter local. 
Consideramos los conjuntos $A_{\alpha}^- = \{x\in A \ : \ x<a\}$ y $A_{\alpha}^+ = \{x\in A \ : \ x>a\}$, $A$ se acumula a la izquierda (res. derecha) de $\alpha$ si $\forall\delta>0 \ ]\alpha-\delta, \alpha[\cap A \not =\emptyset$.
Sea $\alpha\in (A_{\alpha}^-)'$
$$ \lim_{x\rightarrow\alpha -} f(x) = L \Longleftrightarrow
 [ x_n\in A, \ x_n<\alpha \ \forall n\in\N, \ \{x_n\}\rightarrow\alpha \implies \{f(x_n)\}\rightarrow L ] $$

Las caracterizaciones anteriores para límite tienen su versión para límites laterales, en el  límite por la izquierda son sucesiones crecientes y $\alpha-\delta<x<\alpha$ en la caracterización $(\epsilon-\delta)$

Si $A$ se acumula a la izquierda pero no a la derecha de $\alpha\ \ $ $ \lim_{x\rightarrow\alpha} f(x) = L \Longleftrightarrow
 \lim_{x\rightarrow \alpha-} f(x) = L $

Si límites laterales no coinciden, tenemos una discontinuidad de salto, en otro caso
$$ \lim_{x\rightarrow\alpha} f(x) = L \Longleftrightarrow
 \lim_{x\rightarrow\alpha-} f(x) = \lim_{x\rightarrow\alpha+} f(x) = L$$
\section{Límites en el infinito, funciones divergentes}
Sea $f:A\rightarrow\R$ y $A$ no mayorado:
$$ \limx f(x) = L\in\R \Longleftrightarrow
 [x_n\in A \ \forall n\in\N, \ \{x_n\}\rightarrow+\infty \implies \{f(x_n)\} \rightarrow L]$$
$$ \limx f(x) = L \Longleftrightarrow \lim_{y\rightarrow0^+} f(1/y) = L $$
Sea $A$ conjunto no mayorado, $f:A\rightarrow\R$, $L\in\R$, son equivalentes
\begin{itemize}
	\item $\limx f(x) = L$
	\item $\forall\{x_n\}$ de puntos de $A$, creciente y no mayorada, $\{f(x_n)\}\rightarrow L$
	\item $\forall\epsilon>0 \ \exists K\in\R \ : \ x>K \implies |f(x)-L|<\epsilon$
\end{itemize}

\subsection{Funciones divergentes en un punto}
\begin{center}
$ \lim_{x\rightarrow\alpha\in A'} f(x) = +\infty \Longleftrightarrow
[x_n\in A\backslash\{\alpha\} \ \forall n\in\N, \{x_n\}\rightarrow\alpha \implies \{f(x_n)\}\rightarrow+\infty] $
\end{center}

 
$f$ diverge cuando $|f|$ diverge positivamente, son propiedades locales. Son equivalentes
\begin{itemize}
	\item $\{x_n\}\rightarrow\alpha$ monótona de puntos de $A\backslash\{\alpha\} \implies \{f(x_n)\} \rightarrow +\infty$
	\item $\forall K\in\R, \ \exists\epsilon>0 \ : \ x\in A, 0<|x-\alpha|<\delta \implies f(x)>K$
\end{itemize}
Las nociones de divergencia lateral se deducen fácilmente. Si $A$ se acumula a ambos lados de $\alpha$
$$ \lim_{x\rightarrow\alpha} f(x) = +\infty \Longleftrightarrow 
\left\lbrace
\begin{array}{l}
 \lim_{x\rightarrow\alpha-} f(x) = +\infty \\
 \lim_{x\rightarrow\alpha+} f(x) = +\infty \\
\end{array}
\right.
$$

\subsection{Divergencia en el infinito}
La definición por sucesiones es equivalente con $L=+\infty$, solamente cambia la $3^a$ equivalencia
\begin{itemize}
	\item $\forall K\in\R \ \exists m\in\R \ : \ x\in A, \  x>M \implies f(x)>K$
\end{itemize}
$$\lim_{x\rightarrow\alpha} f(x)=\beta \text{ y $g$ es continua en $\beta$}, \ \lim_{x\rightarrow\alpha} (gof)(x) = g(\beta)$$

\section{Derivación}
\begin{center}
$ f'(a) = \lim_{x\rightarrow a} \frac{f(x)-f(a)}{x-a} \hspace{1cm} a\in A\cap A' $
\end{center}

La derivabilidad tiene carácter local e implica la continuidad. 
Las derivadas laterales se denotan por $f'(a-)$ y $f'(a+)$.
Si $f$ es derivable por izquierda y derecha es continua, aunque las derivadas no coincidan. Existe un polinomio $P(x)$ que representa la recta tangente por el punto $a$

$$ P(x) = f(a) + f'(a)(x-a) \ \ \forall x\in\R \hspace{2cm}
\lim_{x\rightarrow a} \frac{f(x)-P(x)}{x-a} = 0$$

Caracterización $(\epsilon-\delta)$ de la derivada
$$ x\in A, \ |x-a|<\delta \implies |f(x)-f(a)-f'(a)(x-a)|\leq \epsilon |x-a| $$
Si $f$ es derivable, $f$ es diferenciable con diferencial de $f$ en $a$,
$ df(a)(h) = f'(a)h \hspace{0.5cm} \forall h\in\R $
$$ f'(a) = \frac{df(a)}{dx} = \lim_{h\rightarrow 0}\frac{f(a+h)-f(a)}{h} $$

La recta secante por los puntos es \  $p_1 = (x_1, y_1), \ p_2 = (x_2, y_2)$ es $\ y-y_1 = \frac{y_2-y_1}{x_2-x_1}(x-x_1)$

La interpretación física de la derivada es la velocidad de variación de una magnitud física.


\section{Reglas de derivación}
\begin{center}
$ (f+g)'(x) = f'(a)+g'(a) \hspace{1cm}
(fg)'(x) = f'(a)g(a) + f(a)g'(a) \hspace{1cm}
\left(\frac{f}{g}\right)'(a) = \frac{f'(a)g(a) - f(a)g'(a)}{g(a)^2}$
\end{center}
\subsection{Regla de la cadena}
$f:A\rightarrow\R$, $g:B\rightarrow\R$ con $f(A)\subset B$ y $a\in A\cap A'$, deben ser $f$ derivable en $a$ y $g$ derivable en $f(a)$
$$ (gof)'(a) = g'(f(a))f'(a) $$
Sea una función $f:A\rightarrow\R$ inyectiva con inversa derivable en $a$, entonces $f(a)\in f(A)$, son equivalentes
\begin{itemize}
	\item $f'(a)\not =0$ y $f^{-1}$ es continua en $f(a)$
	\item $f^{-1}$ es derivable en $f(a)$
\end{itemize}
$$ (f^{-1})'(f(a)) = \frac{1}{f'(a)} $$
$$ f'(x^k) = kx^{k-1} \ \ k\in\N \hspace{1.5cm}
 f'(\sqrt[q]{x}) = \frac{1}{ q(\sqrt[q]{x})^{q-1} }$$

\section{Teorema del valor medio}
$a$ es un extremo relativo cuando $\exists\delta>0 \ : \ ]a-\delta, a+\delta[\subset A \ y \  f(a)\leq f(x) (resp. \geq) \ \forall x\in ]a-\delta, a+\delta[$, hablamos de extremo absoluto cuando se cumple la condición $\forall x\in A$. Un extremo absoluto será relativo si pertenece al interior del conjunto, $A^o$, y en caso de ser derivable $f'(a) = 0$

\textbf{Teorema de Rolle}
Sean $a<b$ y $f\in C([a,b])\cap D(]a,b[)$ con $f(a) = f(b)\implies\exists c\in ]a,b[, \ : \ f(c) =0$
\textbf{Teorema del Valor Medio} 
$a<b$ y $f\in C([a,b])\cap D(]a,b[) \implies\exists c\in ]a, b[ \ : \ f(b)-f(a)=f'(c)(b-a)$

Sea $I$ intervalo no trivial
, $f\in C(I)\cap D(I^o)$.
$$ f'(x)\geq 0 \ \ \forall x\in I^o\Longleftrightarrow f \text{ es creciente} \hspace{1.5cm}
f'(x)\leq 0 \ \ \forall x\in I^o\Longleftrightarrow f \text{ es decreciente} $$
$$ f'(x)\not =0 º \ \forall x\in I^o \implies f \text{ es estrictamente monótona} $$

\textbf{Teorema del valor intermedio para derivadas (Darboux).} Sea $I$ un intervalo no trivial y$f:I\rightarrow\R$, $f\in D(I)$. Entonces $f'(I)$ es un intervalo. Por tanto, $f'$ tiene la propiedad del valor intermedio.

Si $f$ es derivable en $J=[a,b]$, entonces $f'$ no tiene discontinuidades evitables ni de salto y no diverge.
\section{Continuidad uniforme}
Cuando $\delta$ dependa solamente de $\epsilon$ y no de los puntos tomados tendremos continuidad uniforme
$$ \forall\epsilon>0,\ \exists\delta>0\ : \ x,y\in A, \ |y-x|<\delta \implies |f(y)-f(x)|<\epsilon $$
Ejemplo de función continua no uniforme $x^2$. Si $f:A\rightarrow\R$ es uniformemente continua, $\forall \{x_n\},\{y_n\}$ de puntos de $A$ tales que $\{y_n-x_n\}\rightarrow0 \implies \{f(y_n)-f(x_n)\}\rightarrow 0$. $f$ es lipschitziana si existe $M\geq 0$
$$ |f(y)-f(x)|\leq M|y-x| \hspace{0.4cm} \forall x,y\in A\hspace{1cm} 
sup\left\{\frac{|f(y)-f(x)|}{|y-x|} \ :  x,y\in A,  x\not = y \right\} \text{ constante de Lipschitz}$$
Una función es lipschitziana si su derivada está acotada, y lipschitziana $\implies$ uniformemente continua.

La función raíz cuadrada es uniformemente continua pero no lipschitziana.

\subsection{Teorema de Heine}
Sean $a<b$ y $f:[a,b]\rightarrow\R$ continua, entonces $f$ es uniformemente continua.

\section{Integración}
$a,b\in\R$, $a<b$ y $f:[a,b]\rightarrow\R$ continua. 
$\Pi[a,b]$ es el conjunto de las particiones de $[a,b]$, subconjuntos finitos de $[a,b]$ que contienen los extremos. 
Tenemos las sumas inferiores y superiores, $P\in\Pi$
$$ I(f,P) = \sum_{k=1}^{n} (\min f[t_{k-1}, t_k])(t_k-t_{k-1}) \hspace{1.5cm}\{ I(f,P):P\in\Pi[a,b]\} \text{ está mayorado}$$
$$ S(f,P) = \sum_{k=1}^{n} (\max f[t_{k-1}, t_k])(t_k-t_{k-1}) \hspace{1.5cm}\{ S(f,P):P\in\Pi[a,b]\} \text{ está minorado}$$
$$ \sup \{ I(f,P):P\in\Pi[a,b]\} \leq \inf \{ S(f,P):P\in\Pi[a,b]\}  $$
\textbf{Lema} $\forall\epsilon>0, \exists\delta>0 : P\in\Pi[a,b], \ \Delta P<\delta \implies S(f,P)-I(f,P)<\epsilon$

\textbf{Teorema de la integral de Cauchy}
$$ \sup \{ I(f,P):P\in\Pi[a,b]\} = \inf \{ S(f,P):P\in\Pi[a,b]\} = \int_{a}^{b} f(x)dx = \limn \frac{b-a}{n}\sum_{k=1}^{n}f\left(a+\frac{k(b-a)}{n}\right)$$
La integral es lineal, positiva $(\geq0)$ y aditiva, $c\in]a,b[, \ \int_a^b f= \int_a^c f + \int_c^b f$
$$\left|\int_{a}^{b} f(x)dx\right| \leq \int_{a}^{b} |f(x)|dx$$

Si $f\in C[a,b], \ \exists c\in[a,b] : f(c) = \frac{1}{b-a} \int_a^b f(x)dx$
\section{Integral indefinida}
$$ \int_a^b f(x)dx = -\int_b^a f(x)dx \hspace{1.5cm}
\int_a^b f(x)dx = \int_a^c f(x)dx + \int_c^b f(x)dx \ \forall a,b,c\in I, f\in C(I)$$

\subsection{Teorema fundamental del cálculo}
Sea $I$ intervalo no trivial, $f\in C[a,b], a\in I$ y sea
$$ F(x) = \int_a^t f(t) dt \hspace{0.5cm} \forall x\in I\hspace{1.5cm} F\in D(I), \ F'(x) = f(x) \hspace{0.5cm}\forall x\in I$$
$F$ es la integral indefinida de $f$ con origen en $a$.
Cualquier integral indefinida de una función continua en un intervalo no trivial es una primitiva de dicha función. 
Toda función continua en un intervalo no trivial admite primitiva, y entonces tiene la propiedad del valor intermedio.
$F\in C^1(A)$ cuando es derivable con derivada continua en $A$.

\subsection{Regla de Barrow}
Si $I$ es un intervalo no trivial, $f\in C(I)$ y $G$ es una primitiva de $f$
$$ \int_a^b f(x)dx = G(b)-G(a) $$
\subsection{Cambio de variable}
Sea $I$ intervalo no trivial, $f\in C(I), \ a,b\in I$. Sea $J$ intervalo no trivial y $\Phi\in C^1(I)$ con $\Phi(J)\subset I$, y existen $\alpha,\beta\in J : a=\Phi(\alpha),b=\Phi(\beta)$
$$ \int_a^b f(x)dx = \int_{\alpha}^{\beta} f(\Phi(t))\Phi'(t)dt $$

\subsection{Integración por partes}
Sea $I$ un intervalo no trivial y $f,G\in C^1(I), \ \forall x,y\in I$
$$ \int_a^b f(x)G'(x)dx = [f(x)G(x)]_a^b - \int_a^b f'(x)G(x)dx$$
\section{Potencias y logaritmos}
\subsection{Logaritmo}
$$ \forall x\in\R^+ \hspace{1cm}log x = \int_1^x \frac{dt}{t} \hspace{1.5cm}f'(x)=\frac{1}{x} $$
El logaritmo es estrictamente creciente, diverge positivamente en $+\infty$ y negativamente en $0$
$$ log \ (x/y) = log\ x-log\ y\hspace{1.5cm} log(x^n)=nlog\ x \hspace{1.5cm} log\ e=1$$

\subsection{Exponencial}
\begin{center}
$ exp = log^{-1} \hspace{1.5cm} exp:\R\rightarrow\R^+ \hspace{0.3cm} \text{ es biyectiva} $
\end{center}
$$ exp(x-y) = exp\ x/exp\ y \hspace{1.2cm} exp(nx) = (exp\ x)^n \hspace{1.2cm} \lim_{x\rightarrow -\infty} exp\ x = 0\hspace{1.2cm} \lim_{x\rightarrow +\infty} exp\ x= +\infty$$

\subsection{Potencias de exponente real}
\begin{center}
$ a^b = exp(blog\ a) \hspace{0.5cm} \forall b\in\R, \ \forall a\in\R^+ \hspace{1.5cm} a^x\text{ es derivable: }a^xlog \ a$
\end{center}
$$ a^{log_a\ y} = y \ \ \forall y\in\R+\hspace{1.5cm} log_a\ a^x = x\ \ \forall x\in\R \hspace{1.5cm} log_a\ x = \frac{log\ x}{log\ a}$$
$log_a \ x$ con $a>1$ estrictamente creciente, $a<1$ estrictamente decreciente.
\subsection{Escala de infinitos}
$$ \forall p\in\R^+ \ \ \limx \frac{log\ x}{x^p} = \limx \frac{x^p}{e^x} = \lim_{x\rightarrow 0} x^p log\ x = 
\limx x^{1/x} = \lim_{x\rightarrow0} x^x = 0 $$

\subsection{Series armónicas y series de Bertrand}
$$ \sum_{n\geq 1} \frac{1}{n^{\alpha}} \text{ converge si, y sólo si, } \alpha>1 $$
$$ \sum_{n\geq 3} \frac{1}{n^{\alpha}(log\ n)^{\beta}} \text{ serie de Bertrand con exponentes }\alpha,\beta\in\R $$
Converge cuando $\alpha>1$ y diverge cuando $\alpha<1$. Si $\alpha = 1$, converge si, y sólo si, $\beta>1$

\subsection{Equivalencia logarítmica}
Sea $\{x_n\}\rightarrow 1$, $\{y_n\}$, si $L\in\R$, entonces $\limn \ y_n(x_n-1) = L \Longleftrightarrow \limn x_n^{y_n} = e^L$ww
\section{Funciones trigonométricas}
$$\arctan x = \int_0^x \frac{dt}{1+t^2}\hspace{0.5cm} \arctan : \R\rightarrow \left]-\frac{\pi}{2}, \frac{\pi}{2} \right[$$
Función estrictamente creciente, impar $\arctan(-x) = -\arctan \ x$. $\pi = 4\arctan 1$.
La tangente es una función $\pi$-periódica, inversa e la tangente en $]-\pi/2,+\pi/2[$, con $\tan (x+\pi) = \tan \ x$

Definimos $B=\R\backslash\{(2k-1)\pi : k\in\Z\}$
$$ \sin \ x = \frac{2\tan (x/2)}{1+\tan^2 (x/2)} \hspace{0.5cm} \forall x\in B, \hspace{1cm} \sin \ x = 0\hspace{0.5cm} \forall x\in\R\backslash B $$ 
$$ \cos \ x = \frac{1-\tan^2 (x/2)}{1+\tan^2 (x/2)} \hspace{0.5cm} \forall x\in B, \hspace{1cm} \cos \ x = -1\hspace{0.5cm} \forall x\in\R\backslash B $$
$$ \arcsin x = \sin^{-1}x \hspace{0.5cm} \arcsin:[-1, 1] \rightarrow\left[-\frac{\pi}{2}, \frac{\pi}{2}\right]$$
$$ \arccos x = \cos^{-1}x \hspace{0.5cm} \arccos:[-1, 1] \rightarrow\left[-\frac{\pi}{2}, \frac{\pi}{2}\right] $$
$$ \sec x = \frac{1}{\cos x}\hspace{1.5cm}
cosec \ x=\frac{1}{\sin x} \hspace{1.5cm}
cotg \ x =\frac{\cos x}{\sin x}
$$
\subsection{Operaciones}
$$ \sin(x+y) = \sin x\cos y + \cos x\sin y \hspace{2cm}
\cos(x+y) = \cos x\cos y-\sin x\sin y$$
$$ \sin (x+(\pi/2)) = \cos x\hspace{2cm} \cos (x+(\pi/2)) = -\sin x $$
$$ \arcsin x+\arccos x = \frac{\pi}{2} $$
$$ \cos (2x) = \frac{1-\tan^2 x}{1+\tan^x}\hspace{1.5cm} \sin (2x) = \frac{2\tan x}{1+\tan^2 x} $$
$\forall(x,y)\in\R^2\backslash\{(0,0)\}$, existe un único $p\in\R^+$, y un único $t\in]-\pi, \pi[$, tales que $(x,y) = (p\cos t,p\sin t)$
\section{Reglas de l'Hôpital}
\subsection{Teorema del Valor Medio Generalizado}
Sean $a,b\in\R$ con $a<b$ y $f,g\in C([a,b])\cap D(]a,b[)$. Entonces, existe $c\in[a,b]$ verificando
$$ (f(b)-f(a))g'(c) = (g(b)-g(a))f'(c) $$
\subsection{Reglas de l'Hôpital}
Sea $I$ un intervalo no trivial, $a\in I$ y $f,g:I\backslash A\rightarrow\R$ verificando
\begin{center}
Derivables en $I\backslash A$, $g(x)\not = 0 \ \forall x\in I\backslash \{a\}$ y $[\lim_{x\rightarrow a} f(x) = \lim_{x\rightarrow a} g(x) = 0 \text{ ó } \lim_{x\rightarrow a} = \pm\infty \text{ o } \infty]\implies$
\end{center}
$$ \lim_{x\rightarrow a} \frac{f'(x)}{g'(x)}=L\implies \lim_{x\rightarrow a} \frac{f(x)}{g(x)}= L\hspace{1cm} L\in\R\cup\{\pm\infty\}$$
Esta regla puede aplicarse al estudio de límites laterales
\section{Derivadas sucesivas}
$$ (fg)^{(n)}= \sum_{k=0}^{\infty} \binom{n}{k} f^{(n-k)}g^{(k)} $$
Las funciones racionales son de clase $C^{\infty}$. Si dos funciones son de una clase $C^k$, su composición, sus inversas también lo son.

$$f_n(x) = x^{n-1}|x| \hspace{0.5cm}\forall x\in\R \hspace{1cm} f_n(x)\in C^{n-1}(\R)\backslash D^n(\R)$$
$$ g_n(x) = x^{2n}\sin (1/x) \hspace{0.5cm}\forall x\in\R^*\hspace{1cm} g_n(x)\in D^n(\R)\backslash C^n(\R)$$

\section{Fórmula de Taylor}
Sea $I$ intervalo $a\in I^o$ y $f\in D^{n-1}(I)$. Supongmos que $f^{(k)}(a) = 0$ para $1\leq k<n$ y que $f$ es $n$ veces derivable en $a$ con $f^{(n)}(a) \not=0$
\begin{itemize}
	\item Si $n$ es par y $f^{(n)}(a) >0$, $f$ tiene mínimo relativo en $a$, si $f^{(n)}(a) <0$, $f$ tiene un máximo relativo en $a$
	\item Si $n$ es impar, $f$ no tiene extremo relativo en $a$
\end{itemize}
\textbf{Fórmula de Taylor con resto de Lagrange}
Sea $I$ intervalo no trivial y $f\in C^n(I)\cap D^{n+1}(I^o)$ con $n\in\N\cup\{0\}$. Entonces, para cualesquiera $a,x\in I : a\not=x$, existe un $c\in\min\{x,a\}<c<\max\{x,a\}$
$$ R_n[f,a](x) = \frac{f^{(n+1)}(c)}{(n+1)!} (x-a)^{n+1} $$

Serie de Taylor:
$$ \sum_{n\geq0} \frac{f^{(n)}(a)}{n!}(x-a)^n $$
$$ e^x = \sum_{n=0}^{\infty} \frac{x^n}{n!} \hspace{1cm}
\sin x=\sum_{n=0}^{\infty} \frac{\sin (a+(n\pi/2))}{n!} (x-a)^n \hspace{1cm}
\cos x=\sum_{n=0}^{\infty} \frac{cos(a+(n\pi/2))}{n!} (x-a)^n
\hspace{0.4cm}\forall a,x\in\R $$
$$ \log (1+x)\sum_{n=1}^{\infty} \frac{(-1)^{n+1} x^n}{n} \hspace{0.5cm} \forall x\in]-1, 1[\hspace{1.5cm}
\arctan x = \sum_{n=0}^{\infty} \frac{(-1)^n x^{2n+1}}{2n+1}\hspace{0.5cm} 
\forall x\in[-1, 1]$$
\section{Funciones convexas}
Una función será convexa cuando la gráfica de la restricción de $f$ a cualquier intervalo cerrado y acotado queda siempre por debajo del segmento que pasa por $(a,f(a)), (b,f(b))$

Si $I$ es un intervalo no trivial, $f:I\rightarrow\R$ es convexa cuando
$$ f((1-t)x+ty) \leq (1-t)f(x) + tf(y)\hspace{0.5cm}
\forall x,y\in I,\ \forall t\in[0,1] $$

$f$ es cóncava cuando $-f$ es convexa, si es convexa tiene pendiente creciente.

\textbf{Lema de las tres secantes.}

$f:I\rightarrow\R$ una función convexa. Entonces $\forall x_1,x_2,x_3\in I : x_1<x_2<x_3$
$$ \frac{f(x_2)-f(x_1)}{x_2-x_1} \leq 
   \frac{f(x_3)-f(x_1)}{x_3-x_1} \leq 
   \frac{f(x_3)-f(x_2)}{x_3-x_2}$$
   
\textbf{Teorema.}
$f:I\rightarrow\R$ convexa. Entonces $f$ es derivable por la izquierda y por la derecha, y por tanto es continua, $\forall a\in I^o$

\begin{itemize}
	\item $g:I\rightarrow\R$ creciente. Entonces $g$ tiene límite por la izquierda y por la derecha en todo punto $a\in I^o$
	\item $f\in D^1(I)$, son equivalentes
	\begin{itemize}
		\item $f$ es convexa
		\item $f'$ es creciente
		\item $\forall a,x\in I$ se tiene que $f(x)\geq f(a)+f'(a)(x-a)$
	\end{itemize}
\end{itemize}

\begin{itemize}
	\item $f\in C^1(I)\cap D^2(I^o)$. Entonces $f$ es convexa si, y sólo si, $f''(x)\geq 0 \ \forall x\in I^o$
\end{itemize}
\subsection{Ejemplos}
Todo polinomio de primer orden es cóncavo y convexo  (Recíproco verdadero)

La función potencia de exponente $\alpha\in\R$ es convexa cuando $\alpha\leq 0$ o $\alpha\geq1$, y cóncava cuando $0\leq\alpha\leq1$ (en $\R^+$)

La exponencial es una función convexa y el logaritmo es una función cóncava

El $\arctan x$ es una función convexa en $\R^+_0$ y cóncava en $\R^+_0$



%%%%%%%%%%%%%%%%%%%%%%%%%%%%%%%%%%%%%%%%%%%%%%%%%%%%%%%%%%%%%%%%%%%%%%%%%%%%%%%%%%

% % % % % % % % % % % % % % % % % % % % % % % % % % % % % % % % %
%					 Bibliografía
% % % % % % % % % % % % % % % % % % % % % % % % % % % % % % % % %

\end{document}
