\section{Continuidad uniforme}
Cuando $\delta$ dependa solamente de $\epsilon$ y no de los puntos tomados tendremos continuidad uniforme
$$ \forall\epsilon>0,\ \exists\delta>0\ : \ x,y\in A, \ |y-x|<\delta \implies |f(y)-f(x)|<\epsilon $$
Ejemplo de función continua no uniforme $x^2$. Si $f:A\rightarrow\R$ es uniformemente continua, $\forall \{x_n\},\{y_n\}$ de puntos de $A$ tales que $\{y_n-x_n\}\rightarrow0 \implies \{f(y_n)-f(x_n)\}$. $f$ es lipschitziana si existe $M\geq 0$
$$ |f(y)-f(x)|\leq M|y-x| \hspace{0.4cm} \forall x,y\in A\hspace{1cm} 
sup\left\{\frac{|f(y)-f(x)|}{|y-x|} \ :  x,y\in A,  x\not = y \right\} \text{ constante de Lipschitz}$$
Una función es lipschitziana si su derivada está acotada, y lipschitziana $\implies$ uniformemente continua.

La función raíz cuadrada es uniformemente continua pero no lipschitziana.

\subsection{Teorema de Heine}
Sean $a<b$ y $f:[a,b]\rightarrow\R$ continua, entonces $f$ es uniformemente continua.
