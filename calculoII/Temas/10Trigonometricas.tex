\section{Funciones trigonométricas}
$$\arctan x = \int_0^x \frac{dt}{1+t^2}\hspace{0.5cm} \arctan : \R\rightarrow \left]-\frac{\pi}{2}, \frac{\pi}{2} \right[$$
Función estrictamente creciente, impar $\arctan(-x) = -\arctan \ x$. $\pi = 4\arctan 1$.
La tangente es una función $\pi$-periódica, inversa e la tangente en $]-\pi/2,+\pi/2[$, con $\tan (x+\pi) = \tan \ x$

Definimos $B=\R\backslash\{(2k-1)\pi : k\in\Z\}$
$$ \sin \ x = \frac{2\tan (x/2)}{1+\tan^2 (x/2)} \hspace{0.5cm} \forall x\in B, \hspace{1cm} \sin \ x = 0\hspace{0.5cm} \forall x\in\R\backslash B $$ 
$$ \cos \ x = \frac{1-\tan^2 (x/2)}{1+\tan^2 (x/2)} \hspace{0.5cm} \forall x\in B, \hspace{1cm} \cos \ x = -1\hspace{0.5cm} \forall x\in\R\backslash B $$
$$ \arcsin x = \sin^{-1}x \hspace{0.5cm} \arcsin:[-1, 1] \rightarrow\left[-\frac{\pi}{2}, \frac{\pi}{2}\right]$$
$$ \arccos x = \cos^{-1}x \hspace{0.5cm} \arccos:[-1, 1] \rightarrow\left[-\frac{\pi}{2}, \frac{\pi}{2}\right] $$
$$ \sec x = \frac{1}{\cos x}\hspace{1.5cm}
cosec \ x=\frac{1}{\sin x} \hspace{1.5cm}
cotg \ x =\frac{\cos x}{\sin x}
$$
\subsection{Operaciones}
$$ \sin(x+y) = \sin x\cos y + \cos x\sin y \hspace{2cm}
\cos(x+y) = \cos x\cos y-\sin x\sin y$$
$$ \sin (x+(\pi/2)) = \cos x\hspace{2cm} \cos (x+(\pi/2)) = -\sin x $$
$$ \arcsin x+\arccos x = \frac{\pi}{2} $$
$$ \cos (2x) = \frac{1-\tan^2 x}{1+\tan^x}\hspace{1.5cm} \sin (2x) = \frac{2\tan x}{1+\tan^2 x} $$
$\forall(x,y)\in\R^2\backslash\{(0,0)\}$, existe un único $p\in\R^+$, y un único $t\in]-\pi, \pi[$, tales que $(x,y) = (p\cos t,p\sin t)$