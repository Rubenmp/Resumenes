\section{Integral indefinida}
$$ \int_a^b f(x)dx = -\int_b^a f(x)dx \hspace{1.5cm}
\int_a^b f(x)dx = \int_a^c f(x)dx + \int_c^b f(x)dx \ \forall a,b,c\in I, f\in C(I)$$

\subsection{Teorema fundamental del cálculo}
Sea $I$ intervalo no trivial, $f\in C[a,b], a\in I$ y sea
$$ F(x) = \int_a^t f(t) dt \hspace{0.5cm} \forall x\in I\hspace{1.5cm} F\in D(I), \ F'(x) = f(x) \hspace{0.5cm}\forall x\in I$$
$F$ es la integral indefinida de $f$ con origen en $a$.
Cualquier integral indefinida de una función continua en un intervalo no trivial es una primitiva de dicha función. 
Toda función continua en un intervalo no trivial admite primitiva, y entonces tiene la propiedad del valor intermedio.
$F\in C^1(A)$ cuando es derivable con derivada continua en $A$.

\subsection{Regla de Barrow}
Si $I$ es un intervalo no trivial, $f\in C(I)$ y $G$ es una primitiva de $f$
$$ \int_a^b f(x)dx = G(b)-G(a) $$
\subsection{Cambio de variable}
Sea $I$ intervalo no trivial, $f\in C(I), \ a,b\in I$. Sea $J$ intervalo no trivial y $\Phi\in C^1(I)$ con $\Phi(J)\subset I$, y existen $\alpha,\beta\in J : a=\Phi(\alpha),b=\Phi(\beta)$
$$ \int_a^b f(x)dx = \int_{\alpha}^{\beta} f(\Phi(t))\Phi'(t)dt $$

\subsection{Integración por partes}
Sea $I$ un intervalo no trivial y $f,G\in C^1(I), \ \forall x,y\in I$
$$ \int_a^b f(x)G'(x)dx = [f(x)G(x)]_a^b - \int_a^b f'(x)G(x)dx$$