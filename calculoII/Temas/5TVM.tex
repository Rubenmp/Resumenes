\section{Teorema del valor medio}
$a$ es un extremo relativo cuando $\exists\delta>0 \ : \ ]a-\delta, a+\delta[\subset A \ y \  f(a)\leq f(x) (resp. \geq) \ \forall x\in ]a-\delta, a+\delta[$, hablamos de extremo absoluto cuando se cumple la condición $\forall x\in A$. Un extremo absoluto será relativo si pertenece al interior del conjunto, $A^o$, y en caso de ser derivable $f'(a) = 0$

\textbf{Teorema de Rolle}
Sean $a<b$ y $f\in C([a,b])\cap D(]a,b[)$ con $f(a) = f(b)\implies\exists c\in ]a,b[, \ : \ f(c) =0$
\textbf{Teorema del Valor Medio} 
$a<b$ y $f\in C([a,b])\cap D(]a,b[) \implies\exists c\in ]a, b[ \ : \ f(b)-f(a)=f'(c)(b-a)$

Sea $I$ intervalo no trivial
, $f\in C(I)\cap D(I^o)$.
$$ f'(x)\geq 0 \ \ \forall x\in I^o\Longleftrightarrow f \text{ es creciente} \hspace{1.5cm}
f'(x)\leq 0 \ \ \forall x\in I^o\Longleftrightarrow f \text{ es decreciente} $$
$$ f'(x)\not =0 º \ \forall x\in I^o \implies f \text{ es estrictamente monótona} $$

\textbf{Teorema del valor intermedio para derivadas (Darboux).} Sea $I$ un intervalo no trivial y$f:I\rightarrow\R$, $f\in D(I)$. Entonces $f'(I)$ es un intervalo. Por tanto, $f'$ tiene la propiedad del valor intermedio.

Si $f$ es derivable en $J=[a,b]$, entonces $f'$ no tiene discontinuidades evitables ni de salto y no diverge.